\documentclass{article}

\usepackage{environ}
\usepackage{etoolbox}

\usepackage{tikz}
  \usetikzlibrary{arrows,calc,fit,patterns,plotmarks,shapes.geometric,shapes.misc,shapes.symbols,shapes.arrows,shapes.callouts, shapes.multipart, shapes.gates.logic.US,shapes.gates.logic.IEC, er, automata,backgrounds,chains,topaths,trees,petri,mindmap,matrix, calendar, folding,fadings,through,positioning,scopes,decorations.fractals,decorations.shapes,decorations.text,decorations.pathmorphing,decorations.pathreplacing,decorations.footprints,decorations.markings,shadows,babel}%Charge toutes les librairies de Tikz


\let\oldusetikzlibrary\usetikzlibrary
\renewcommand{\usetikzlibrary}[1]{%
\ifstrequal{#1}{bending}{}{\oldusetikzlibrary{#1}}%
}

\usepackage{scratch3}


\RequirePackage{pgf}
\RequirePackage{pgfplots}
  \pgfplotsset{compat=1.18}
% \tcbuselibrary{skins,breakable}


\NewEnviron{scratch*}[1][]{
  \begin{scratch}[#1]
  \renewarrow
  \BODY
  \end{scratch}
}

\newcommand{\renewarrow}{
\renewcommand{\turnleft}{%
\tikz[baseline=.25ex,x=6.5ex,y=6.5ex]
\draw[-{Triangle[angle=110:0.6ex 0.15ex]}, line width=.3333ex] (0,0) arc[start angle=-80, end angle=180, radius=1ex];
}
\renewcommand{\turnright}{%
\tikz[baseline=.25ex,x=6.5ex,y=6.5ex,xscale=-1]
\draw[-{Triangle[angle=110:0.6ex 0.15ex]},line width=.3333ex](0,0) arc[start angle=-80,end angle=180,radius=1ex];
}
}




\begin{document}

\begin{scratch*}[print]

\blockmove{\turnleft test}
\blockmove{\turnright test}

\end{scratch*}


\end{document}


% \documentclass{scrartcl}
% \usepackage[utf8]{inputenc}
% \usepackage[T1]{fontenc}
% \usepackage{lipsum}  % Package utilisé pour générer du faux texte (lipsum)
% \usepackage{fancyhdr}

% % Configuration fancyhdr
% \pagestyle{fancy}
% \fancyhead{}  % Effacer les en-têtes précédemment définis
% \fancyhead[R]{Mon En-Tête}  % En-tête à droite
% \renewcommand{\headrulewidth}{0.4pt}  % Épaisseur de la ligne de l'en-tête

% \begin{document}

% \title{Mon Document avec En-Tête}
% \author{Moi}
% \date{\today}

% \maketitle

% \section{Section 1}
% \lipsum[1-10]  % Génère du faux texte

% \end{document}










% \documentclass{article}

% \usepackage[french]{babel}
% \usepackage{lipsum}
% \usepackage[svgnames]{xcolor}
% \usepackage{showframe}
% \usepackage{geometry}


% \usepackage{fontspec}
% \usepackage{anyfontsize}
% \usepackage[french]{babel}
% \usepackage{unicode-math}
% \usepackage{enumitem}
% \usepackage{tasks}

% %Formatages divers
% % \usepackage{fontawesome5}
% % \usepackage[minimal]{lua-ul}
% % \usepackage{soul}
% % \usepackage{tcolorbox}
% %   \tcbuselibrary{skins,breakable}
% % \usepackage{environ}

% % \usepackage{tikz}
% % \usepackage{calc}
% % \usepackage



% \RequirePackage{showframe}%A enlever




% \AddToHook{shipout/before}{TEST\newpage}


% \begin{document}

% \lipsum[1-25]


% \end{document}




% \documentclass{article}

% \newcommand{\test}[1]{#1}
% \usepackage{luacode}
% \usepackage{tcolorbox}



% \begin{document}

% \begin{luacode*}
% tex.sprint([[\begin{tcolorbox}[colback=red,
% colframe=red]salut\end{tcolorbox} ]])
% \end{luacode*}


% \test{Coucou}
% % \important

% \begin{test}
% Salut les petits potes
% \end{test}
% \end{document}



% \usepackage{luacode}
% \usepackage{environ}

% \newif\ifvisible

% \begin{luacode}
% PARAMETRES = {}

% PARAMETRES.visible = true

% function VerificationVisible(FLAG,PARAMETRES)
%   local VISIBLE = tostring(PARAMETRES.visible)

%   if FLAG == VISIBLE then
%     tex.sprint('\\visibletrue')
%   else
%     tex.sprint('\\visiblefalse')
%   end
% end

% \end{luacode}

% \newcommand{\visible}[2][true]{%
%   \directlua{VerificationVisible('#1',PARAMETRES)}
%   \ifvisible
%     #2
%   \fi
% }


% \NewEnviron{visibleu}[1][true]{%
%   \directlua{VerificationVisible('#1',PARAMETRES)}
%   \ifvisible
%     \BODY
%   \fi
% }




% \newenvironment{visible}[1][true]{\directlua{VerificationVisible('#1',PARAMETRES)}\ifvisible \makeatletter\def\monenv@code{} \collect@body\monenv@code\makeatother \fi}{}

% \newenvironment{visible}[1][true]{%
%   \directlua{VerificationVisible('#1',PARAMETRES)}%
%   \ifvisible
%     \makeatletter
%     \def\monenv@code{} % Initialisation de \monenv@code
%     \collect@body\monenv@code
%     \makeatother
%   \fi
% }{}

% \newenvironment{visible}[1][true]{%
%   \directlua{VerificationVisible('#1', PARAMETRES)}%
%   \ifvisible
%     \expandafter\monenv@start
%   \else
%     % Ne rien faire si \ifvisible est faux
%   \fi
% }{%
%   \ifvisible
%     \monenv@end
%   \fi
% }

% \makeatletter
% \def\monenv@start#1\monenv@end{%
%   \def\monenv@code{#1}%
%   % Autres actions à effectuer au début de l'environnement
% }

% \def\monenv@end{%
%   % Actions à effectuer à la fin de l'environnement
%   \collect@body\monenv@code
% }
% \makeatother




\begin{document}

\visible{salut}

\begin{visibleu}[true]

visible true
\end{visibleu}

\begin{visibleu}[false]

visible false
\end{visibleu}


\end{document}





% \documentclass{article}
% \usepackage{environ}
% \usepackage{showframe}
% % \NewEnviron{monenv}{
% %   \expandafter\splitbody\BODY\separation\relax
% % }

% \long\def\splitbody#1\separation#2\relax{
%   % Traitement de la première partie (#1) avant la séparation
%   \textbf{Partie 1:} #1\par
%   % Traitement de la deuxième partie (#2) après la séparation
%   \textbf{Partie 2:} #2\par
% }



% \begin{document}

% \newif\ifswitch

% \switchtrue

% \newenvironment{monenv}{\ifswitch}{\fi}

% \begin{monenv}

% test test 

% \else

% test 

% \end{monenv}



% \splitbody 

% \begin{minipage}{\linewidth}
%   test
% \end{minipage}


% \fbox{salut}


% \separation 


% les gars

% \relax

% % \begin{monenv}
% %   Contenu 1
% %   \separation

% %   Contenu 2
% % \end{monenv}

% \end{document}














% \PassOptionsToPackage{table, svgnames}{xcolor}

% \documentclass{scrartcl}

% \usepackage{showframe}
% % \usepackage{luacode}
% % \usepackage{geometry}
% \usepackage[french]{babel}
% \usepackage{lipsum}
% % \usepackage{atbegshi}

% % \usepackage[svgnames]{xcolor}

% % \RequirePackage{pstricks}
% % \RequirePackage{pst-plot}
% % \RequirePackage{pst-eucl}
% % \usepackage{ProfCollege}


% \usepackage{xcolor}
% \usepackage{tcolorbox}
%   \tcbuselibrary{skins,breakable}

% \tcbset{
%   mystyleabove/.style={
%     fonttitle=\bfseries\large,
%     coltitle=blue,
%     colback=yellow!10,
%   },
%   mystylebelow/.style={
%     fonttitle=\itshape\large,
%     coltitle=red,
%     colback=green!10,
%   },
% }

% \newtcolorbox{mytwopartbox}{
%   enhanced,
%   colframe=black,
%   colback=white,
%   title={Partie supérieure},
%   mystyleabove, % Style pour la partie supérieure
%   %before lower=\tcblower, % Marque le passage à la partie inférieure
%   before upper={\tcbtitle\par}, % Applique le style avant la partie supérieure
%   after upper={\par}, % Saut de ligne après la partie supérieure
%   mystylebelow, % Style pour la partie inférieure
% }


% % \newtcolorbox{invisiblebox}{
% %   enhanced,
% %   colback=white, % Couleur de fond
% %   colframe=white, % Couleur du cadre
% %   arc=0mm, % Rayon des coins
% %   boxrule=0mm, % Épaisseur du cadre
% %   top=0mm, % Marge supérieure
% %   bottom=0mm, % Marge inférieure
% %   left=0mm, % Indentation à la valeur de \parindent
% %   right=0mm, % Marge droite
% %   boxsep=0mm, % Padding interne
% %   before skip=\baselineskip -0.9em, % Espace avant
% %   after skip=\baselineskip - 0.9em, % Espace après
% %   before upper={}, % Contenu en haut
% %   after upper={}, % Espace entre les parties
% %   before lower={}, % Contenu en bas
% % }


% % \newcommand{\maCommande}{\textcolor{Red}{TEST} }

% % \AtBeginShipout{\maCommande}
% % \AtBeginDocument{\maCommande}

% \begin{document}

% \begin{mytwopartbox}[upperbox=ignored]
%   Ceci est le contenu de la partie supérieure.
%   \tcblower
%   Ceci est le contenu de la partie inférieure.
% \end{mytwopartbox}


% \lipsum[1]
% % \begin{invisiblebox}
% %   \lipsum[1]
% % \end{invisiblebox}\par

% \lipsum[1]
% % \lipsum[1]\par
% % \lipsum[1]\par
% % \begin{invisiblebox}
% %  \lipsum[1]
% % \end{invisiblebox}
% \end{document}
