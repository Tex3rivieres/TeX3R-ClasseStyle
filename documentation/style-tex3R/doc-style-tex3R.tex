\documentclass{classe-tex3R}
\usepackage{style-tex3R}
\usepackage{dirtree}

\colorlet{colorBasic}{SlateGrey}
\colorlet{colorBracket}{DarkViolet}
\colorlet{colorKeyword}{DarkBlue}
\colorlet{colorEnvironment}{LightSeaGreen}
\colorlet{colorComment}{Green}
\colorlet{colorString}{DarkOrange}
\colorlet{colorChange}{Red}
\colorlet{colorMath}{Teal}
\colorlet{colorVariable}{LightSkyBlue}
\colorlet{colorVariableDeux}{Teal}
\colorlet{colorBoolean}{Indigo}
\colorlet{colorFunction}{DarkBlue}

\newcommand{\nocontentsline}[3]{}
\newcommand{\tocless}[2]{\bgroup\let\addcontentsline=\nocontentsline#1{#2}\egroup}

\lstset{
  %Réglages généraux
  backgroundcolor=\color{gray!20!white},
  basicstyle=\ttfamily\small\color{colorBasic},
  columns=fullflexible,
  aboveskip=\baselineskip,
  belowskip=0pt,
  breaklines=true,
  %Keywords
  emph={[1]
    begin,
    end,
    documentclass,
    usepackage,
    definirchapitre,
    definirniveau,
    parametrage,
    bareme,
    competence,
    difficulte,
    source,
    theme,
    montitre,
    important,
    tailletexte,
    partie,
    souspartie,
    chapitre,
    visiblecmd,
    saut,
    lignes,
    ifdiapo,
    iffiche,
    ifheader,
    ifprint,
    ifactivite,
    ifbasique,
    ifbilan,
    ifcorrige,
    ifcours,
    ifTD,
    ifflash,
    ifDM,
    ifDS,
    ifinterro,
    ifcorrection,
    ifenonce,
    ifvisible,
    ifbareme,
    ifdifficulte,
    ifcompetence,
    ifsource,
    iftheme,
    ifstretch,
    ifsubfile,
    else,
    fi,
    newcounter,
    setcounter,
    stepcounter,
    thecompteurexercice,
  }, 
  emphstyle={[1]\color{colorKeyword}},
  %Environnements
  emph={[2]
    environnement,
    luacode,
    fiche,
    document,
    application,
    convention,
    definition,
    exemple,
    methode,
    propriete,
    preuve,
    remarque,
    enonce,
    correction,
    visible
  }, 
  emphstyle={[2]\color{colorEnvironment}},
  %Variable lua
  emph={[3]
  Type,
  Impression,
  Header,
  Taille,
  Stretch,
  Correction,
  Enonce,
  Visible,
  Competence,
  Bareme,
  Difficulte,
  Source,
  Theme
  },
  emphstyle={[3]\color{colorVariable}},
  %Booléens lua
  emph={[4]
  true,
  false,
  nil
  },
  emphstyle={[4]\color{colorBoolean}},
  %Fonctions lua
  emph={[5]
  mesParametres
  },
  emphstyle={[4]\color{colorFunction}},
  %Variables lua
  emph={[6]
  PAGE,
  },
  emphstyle={[6]\color{colorVariable}},
  %Variables deux lua
  emph={[7]
  enumi,
  enumii,
  itemi,
  itemii,
  },
  emphstyle={[7]\color{colorVariableDeux}},
  %Literate
  literate={
    % {\{}{{\textcolor{colorBracket}{\{}}}1
    % {\}}{{\textcolor{colorBracket}{\}}}}1
    % {[}{{\textcolor{colorBracket}{[}}}1
    % {]}{{\textcolor{colorBracket}{]}}}1
    % {[[}{{\textcolor{colorString}{[[ }}}1
    % {]]}{{\textcolor{colorString}{]]}}}1
    % {\\}{{\textcolor{colorKeyword}{\textbackslash}}}1
    % {0}{{$\cancel{0}$}}1
    },
  %Comment
  comment=[l]{--},
  morecomment=[l]{\%},
  commentstyle=\color{colorComment},
  %Delimiters
  moredelim=[is][\color{colorChange}]{@}{@},
  moredelim=[s][\color{colorMath}]{\$}{\$},
  moredelim=[is][\color{colorString}]{||}{||},
  moredelim=[s][\color{colorString}]{[[}{]]},
  moredelim=[is][\color{colorBasic}]{//}{//},
  moredelim=[is][\color{colorComment}]{/*}{/*},
  }

\title{\bfseries style-tex3R\par}
\author{Vincent Crombez \& Frédéric Léothaud}
\date{}

\cfoot{\pagemark}
\setlength{\parskip}{\baselineskip}
\setcounter{tocdepth}{3}
\renewcommand{\sectionformat}{\LARGE\textbf{\thesection~}}
  \setkomafont{section}{\normalfont\LARGE\color{Black}\bfseries}
\renewcommand{\subsectionformat}{\Large\textbf{\thesubsection~}}
  \setkomafont{subsection}{\normalfont\Large\color{Black}\bfseries}
\renewcommand{\subsubsectionformat}{\large\textbf{\thesubsubsection~}}
  \setkomafont{subsubsection}{\normalfont\large\color{Black}\bfseries}


\begin{document}

\maketitle

\newpage

\tableofcontents

\newpage

\phantomsection
\addcontentsline{toc}{section}{Introduction}
\section*{Introduction}

La \texttt{style-tex3R} fait partie d'un projet comprenant :
%
\begin{itemize}
  \item \href{https://github.com/Tex3rivieres/TeX3R-ClasseStyle}{Une classe, un style et des templates de documents pratiques}
  \item \href{https://github.com/Tex3rivieres/TeX3R-Portable}{Un environnement LaTeX portable}
  \item \href{https://github.com/Tex3rivieres/TeX3R-Workshop}{Une extension (basée sur \texttt{latex-workshop})}
\end{itemize}

Il est préférable d'aller d'abord lire la documentation de \texttt{classe-tex3R} qui explique l'ensemble des commandes disponibles. 

L'objectif de ce guide est de guider la personnalisation du style à appliquer aux environnements et commandes de la \texttt{classe-tex3R}, afin d'avoir une apparence propre selon les goûts de chacun, tout en gardant la possibilité de transférer son document à un autre utilisateur de la \texttt{classe-tex3R}, qui lui le compilera avec le style de son choix.

\newpage

\section{Créer un nouveau style}

Afin d'utiliser son style personnalisé, il suffit de créer une copie du fichier \texttt{style-tex3R.sty} dans le même dossier, de le renommer \texttt{monstyle.sty} et de travailler sur ce dernier. Le fichier \texttt{style-tex3R} se trouve dans l'emplacement suivant :

\saut{ligne} %#1=fiche/diapo

\dirtree{%
.1 TeX3R-ClasseStyle.
  .2 documentation.
  .2 templates.
  .2 tex.
    .3 fonts.
    .3 lualatex.
      .4 tex3R.
        .5 {classe-tex3R.cls}.
        .5 {style-tex3R.sty}.
.2 {.gitignore}.
.2 {README.md}.
}

Une fois le fichier \texttt{monstyle.sty} créé il est nécessaire d'ouvrir MiKTeX Console et de faire Tasks > Refresh file name database.

Ensuite, pour appeler \texttt{monstyle} dans un nouveau document, il suffit de changer le préambule comme suit :

\begin{lstlisting}
\documentclass[fiche]{classe-tex3R}
\usepackage{monstyle}

\begin{document}

%Le texte du document

\end{document}
\end{lstlisting}

\section{Principe de modification}

\texttt{classe-tex3R} et \texttt{style-tex3R} sont codés majoritairement en \texttt{lua}. L'idée derrière cela est d'avoir une part minimale de choses à modifier dans le style, dans des chaînes de caractères \texttt{lua}, qui seront ensuite insérées et utilisées par les commandes de la classe. Ceci étant, les choses pouvant être modifiées sont toujours entre \texttt{[[ ]]} ou entre \texttt{' '} qui sont les délimiteurs des chaînes de caractères en \texttt{lua}. 

Cette expérience simplifiée évite d'avoir à aller voir dans de nombreuses documentations pour connaître la syntaxe exacte pour chaque paramètre. 

Par la suite, nous détaillerons chacune des fonctions \texttt{lua}, et à quoi correspondent les différentes parties modifiables.

\section{FormatUtilisateur}

\adjustbox{valign=t}{\begin{minipage}{0.48\linewidth}%
\begin{lstlisting}
PAGE.enumi = ||[[ \textbf{\arabic*.~} ]]||
PAGE.enumii = [[ \textbf{\alph*.~} ]]
PAGE.itemi = [[ \textbullet ]]
PAGE.itemii = [[ $\circ$ ]]
\end{lstlisting}
\end{minipage}}\hfill%
\adjustbox{valign=t}{\begin{minipage}{0.48\linewidth}%
  
\end{minipage}}%


\end{document}