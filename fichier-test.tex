\documentclass{classe-tex3R-2-1}


\usepackage{style-tex3R-2-1}



\begin{luacode}
PARAMETRES = {}

PARAMETRES.format = 'fiche'
PARAMETRES.type = 'interro'
PARAMETRES.enonce = false
PARAMETRES.correction = true
PARAMETRES.theme = true
PARAMETRES.cache = true
PARAMETRES.header = true
PARAMETRES.print = false
-- PARAMETRES.taille = '50pt'
-- PARAMETRES
\end{luacode}

\rivparametrage








% \settasks{
%   % label-width = 1.5em,
%   label = \labelitemi,
%   % item-indent = 1.5em,
% }

%Apparence de l'enumerate de tasks
% \DeclareInstance{tasks}{itemize}{default}
%   {
%     label=\bfseries\arabic*.,
%     label-width=1.125em
%   }

% Récupérer les valeurs de leftmargin pour itemize et enumerate
% Récupérer les valeurs de leftmargin pour itemize et enumerate



\usepackage{scratch3}

% Nouveau underline soulignant de 2pts (en noir)
\newunderlinetype\beginUnderlineTwoPts{\leaders\hbox{\rule[-0.5pt]{0pt}{2pt}}}

% Commande pour utiliser le nouveau soulignement
\NewDocumentCommand\underlineTwoPts{+m}{{\beginUnderlineTwoPts#1}}

\begin{document}

\tracelignes{5}


\titre{\faCloud}{TEst\hfill \mdseries test}


% \underWavy{Texte avec un soulignement ondulé.}
% \underlineTwoPts{Texte souligné avec une épaisseur de 2pts.}



\ul{test \hfill test}

% \titre{}{}

\begin{enumerate}
  \item test
  \item test
  \begin{enumerate}
    \item test
    \item test
  \end{enumerate}
\end{enumerate}

% \begin{itemize}
%   \item test
%   \item test
%   \begin{itemize}
%     \item test
%     \item test
%   \end{itemize}
% \end{itemize}


\begin{tasks}[style=enumerate]
  \task test
  \task test
\end{tasks}


$5x+8 $

$\mathcal{ABCDEFGHIJK}$

\difficulte{5}

\rivdifficulte


\begin{scratch}
  \blockmove{Salut les petits potes}
\end{scratch}

% \multido{\i=0+1}{\thedifficulte}{\faStar}

% \begin{luacode*}
%   tex.sprint(PAGE.width)
% \end{luacode*}


\ifdiapo \textcolor{green}{diapo}\par \else \hfill \textcolor{red}{no} \textcolor{red}{diapo}\par \fi
\iffiche \textcolor{green}{fiche}\par \else \hfill \textcolor{red}{no} \textcolor{red}{fiche}\par \fi
\ifheader \textcolor{green}{header}\par \else \hfill \textcolor{red}{no} \textcolor{red}{header}\par \fi

% \bigskip

\ifenonce \textcolor{green}{enonce}\par \else \hfill \textcolor{red}{no} \textcolor{red}{enonce}\par \fi
\ifcorrection \textcolor{green}{correction}\par \else \hfill \textcolor{red}{no} \textcolor{red}{correction}\par \fi

% \bigskip


\iftheme \textcolor{green}{theme}\par \else \hfill \textcolor{red}{no} \textcolor{red}{theme}\par \fi
\ifcache \textcolor{green}{cache}\par \else \hfill \textcolor{red}{no} \textcolor{red}{cache}\par \fi
\ifdifficulte \textcolor{green}{difficulte}\par \else \hfill \textcolor{red}{no} \textcolor{red}{difficulte}\par \fi
\ifcompetence \textcolor{green}{competence}\par \else \hfill \textcolor{red}{no} \textcolor{red}{competence}\par \fi
\ifsource \textcolor{green}{source}\par \else \hfill \textcolor{red}{no} \textcolor{red}{source}\par \fi


\bigskip

\ifactivite \textcolor{green}{activite}\par \else \hfill \textcolor{red}{no} \textcolor{red}{activite}\par \fi
\ifbasique \textcolor{green}{basique}\par \else \hfill \textcolor{red}{no} \textcolor{red}{basique}\par \fi
\ifcorrige \textcolor{green}{corrige}\par \else \hfill \textcolor{red}{no} \textcolor{red}{corrige}\par \fi
\ifcours \textcolor{green}{cours}\par \else \hfill \textcolor{red}{no} \textcolor{red}{cours}\par \fi
\ifTD \textcolor{green}{TD}\par \else \hfill \textcolor{red}{no} \textcolor{red}{TD}\par \fi
\ifflash \textcolor{green}{flash}\par \else \hfill \textcolor{red}{no} \textcolor{red}{flash}\par \fi
\ifDM \textcolor{green}{DM}\par \else \hfill \textcolor{red}{no} \textcolor{red}{DM}\par \fi
\ifDS \textcolor{green}{DS}\par \else \hfill \textcolor{red}{no} \textcolor{red}{DS}\par \fi
\ifinterro \textcolor{green}{interro}\par \else \hfill \textcolor{red}{no} \textcolor{red}{interro}\par \fi

\bigskip

\ifprint \textcolor{green}{print}\par \else \hfill \textcolor{red}{no} \textcolor{red}{print}\par \fi

%\ifFirstCompile \textcolor{green}{FirstCompile}\par \else \hfill \textcolor{red}{no} \textcolor{red}{FirstCompile}\par \fi

% \bigskip

% 23 \iflignes \textcolor{green}{lignes}\par \else \hfill \textcolor{red}{no} \textcolor{red}{lignes}\par \fi
% / \ifseyes \textcolor{green}{seyes}\par \else \hfill \textcolor{red}{no} \textcolor{red}{seyes}\par \fi
% 101 \ifsurligne \textcolor{green}{surligne}\par \else \hfill \textcolor{red}{no} \textcolor{red}{surligne}\par \fi
% 83 \ifcarreaux \textcolor{green}{carreaux}\par \else \hfill \textcolor{red}{no} \textcolor{red}{carreaux}\par \fi
% 73 \iflignesetoile \textcolor{green}{lignesetoile}\par \else \hfill \textcolor{red}{no} \textcolor{red}{lignesetoile}\par \fi
% / \ifseyesetoile \textcolor{green}{seyesetoile}\par \else \hfill \textcolor{red}{no} \textcolor{red}{seyesetoile}\par \fi
% 89 \ifcarreauxetoile \textcolor{green}{carreauxetoile}\par \else \hfill \textcolor{red}{no} \textcolor{red}{carreauxetoile}\par \fi

% \bigskip

% ? \ifaffichagetitre \textcolor{green}{affichagetitre}\par \else \hfill \textcolor{red}{no} \textcolor{red}{affichagetitre}\par \fi
% / \ifmegadoctoc \textcolor{green}{megadoctoc}\par \else \hfill \textcolor{red}{no} \textcolor{red}{megadoctoc}\par \fi
% / \ifmegadoc \textcolor{green}{megadoc}\par \else \hfill \textcolor{red}{no} \textcolor{red}{megadoc}\par \fi



\end{document}