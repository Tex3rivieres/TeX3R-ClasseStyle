%%%%%%%%%%%%%%%%%%%%%%%%%%%%%%%%%%%%%%%%%%%%%%%%%%%%%%%%%%%%%%%%%
\documentclass{classe-tex3R}
\usepackage{style-tex3R}

\definirchapitre{5}{Grandeurs proportionnelles}
\definirniveau{6}

\begin{document}

\begin{luacode}
mesParametres('cours')
\end{luacode}
\parametrage

%%%%%%%%%%%%%%%%
%%%% ÉNONCÉ %%%%
%%%%%%%%%%%%%%%%

\competence{Modéliser}
\begin{enonce}
  Ceci est l'énoncé :

  \begin{tasks}[style=itemize](3)
    \task Task1
    \task Task2
    \task Task3
  \end{tasks}
\end{enonce}

\textcolor{rouge}{Salut}

\newpage

\competence{Modéliser}
\begin{enonce}
  Ceci est l'énoncé :

  \begin{tasks}[style=itemize](3)
    \task Task1
    \task Task2
    \task Task3
  \end{tasks}
\end{enonce}

\textcolor{rouge}{Salut}


\begin{luacode}
mesParametres('TD')
\end{luacode}
\parametrage

\competence{Modéliser}
\begin{enonce}
  Ceci est l'énoncé :

  \begin{tasks}[style=itemize](3)
    \task Task1
    \task Task2
    \task Task3
  \end{tasks}
\end{enonce}

\textcolor{rouge}{Salut}

\newpage

\competence{Modéliser}
\begin{enonce}
  Ceci est l'énoncé :

  \begin{tasks}[style=itemize](3)
    \task Task1
    \task Task2
    \task Task3
  \end{tasks}
\end{enonce}

\textcolor{rouge}{Salut}

\begin{luacode}
mesParametres('cours')
\end{luacode}
\parametrage

%%%%%%%%%%%%%%%%
%%%% ÉNONCÉ %%%%
%%%%%%%%%%%%%%%%

\competence{Modéliser}
\begin{enonce}
  Ceci est l'énoncé :

  \begin{tasks}[style=itemize](3)
    \task Task1
    \task Task2
    \task Task3
  \end{tasks}
\end{enonce}

\textcolor{rouge}{Salut}

\newpage

\competence{Modéliser}
\begin{enonce}
  Ceci est l'énoncé :

  \begin{tasks}[style=itemize](3)
    \task Task1
    \task Task2
    \task Task3
  \end{tasks}
\end{enonce}

\textcolor{rouge}{Salut}


\end{document}