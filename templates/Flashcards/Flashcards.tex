\documentclass{classe-tex3R}
\usepackage{style-tex3R}
\parametrage
\usepackage{flashcards-tex3R}

\begin{document}


%%% DÉBUT EXERCICE %%%
\difficulte{-1} %#1=nombre

\begin{enonce}
  \begin{center}
    \includegraphics[width=0.5\linewidth]{example-image-a} 
  \end{center} 

  \medskip

  \textbf{Page de garde}
\end{enonce}

\begin{correction}
  \begin{enumerate}
    \item Ceci
    \item est
    \item la
    \item consigne
  \end{enumerate}
\end{correction}
%%% FIN EXERCICE %%%

%%% DÉBUT EXERCICE %%%
\difficulte{1} %#1=nombre

\begin{enonce}

\begin{tikzpicture}[scale=1,rotate=0]
  \tkzDefPoints{0/0/A, 4/0/B}
  \tkzDrawSegments(A,B)
\end{tikzpicture}


\end{enonce}

\begin{correction}
  Une droite
\end{correction}
%%% FIN EXERCICE %%%

%%% DÉBUT EXERCICE %%%
\difficulte{1} %#1=nombre

\begin{enonce}
  \LARGE $4\times 8$
\end{enonce}

\begin{correction}
  \LARGE $32$
\end{correction}
%%% FIN EXERCICE %%%

%%% DÉBUT EXERCICE %%%
  \begin{luacode*}
    math.randomseed(1) 
    local a = math.random(1,10)
    local b = math.random(1,10)
    local c = a*b
    enonce = "\\LARGE$" .. a .. "\\times" .. b .. "$"
    correction = "\\LARGE$" .. c .. "$"
    table.insert(ENONCE,enonce)
    table.insert(CORRECTION, correction)
    table.insert(DIFFICULTE,"1")
  \end{luacode*}
%%% DÉBUT EXERCICE %%%

\difficulte{2} %#1=nombre

\begin{enonce}
  énoncé 5
\end{enonce}

\begin{correction}
  correction 5
\end{correction}
%%% FIN EXERCICE %%%

%%% DÉBUT EXERCICE %%%
\difficulte{2} %#1=nombre

\begin{enonce}
  énoncé 6
\end{enonce}

\begin{correction}
  correction 6
\end{correction}
%%% FIN EXERCICE %%%

%%% DÉBUT EXERCICE %%%
\difficulte{3} %#1=nombre

\begin{enonce}
  énoncé 7
\end{enonce}

\begin{correction}
  correction 7
\end{correction}
%%% FIN EXERCICE %%%

%%% DÉBUT EXERCICE %%%
\difficulte{4} %#1=nombre

\begin{enonce}
  énoncé 8 
\end{enonce}

\begin{correction}
% \difficultefalse 
  correction 8
\end{correction}
%%% FIN EXERCICE %%%

%%% DÉBUT EXERCICE %%%
\difficulte{-1} %#1=nombre

\begin{enonce}
  Ceci est la page de remerciements
\end{enonce}

\begin{correction}
% \difficultefalse 
  Ceci est l'arrière du livret.
\end{correction}
%%% FIN EXERCICE %%%


\end{document}