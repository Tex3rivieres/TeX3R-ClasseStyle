\documentclass{classe-tex3R}

\usepackage{style-tex3R}
\usepackage{lastpage}

%Commande donnant le mois et l'année de la compilation
\newcommand{\monthyear}{\ifcase\month\or
  Janvier\or Février\or Mars\or Avril\or Mai\or Juin\or
  Juillet\or Août\or Septembre\or Octobre\or Novembre\or Décembre\fi
  \space\number\year
}

%Commande tableau brevet permettant d'afficher le total des points à la deuxième compilation dans la page de garde
\newcommand{\tableaubrevet}{}
\newwrite\myauxbrevet

\IfFileExists{monbrevet.aux}{%
  \renewcommand{\tableaubrevet}{\input{monbrevet.aux}}%
}{
  \immediate\openout\myauxbrevet=monbrevet.aux\relax
  \immediate\write\myauxbrevet{\string\begin{tabular}{|c|c|}}\relax
}

\renewcommand{\bareme}[1]{
  \renewcommand{\contenubareme}{#1 points}
    \stepcounter{compteurbareme}
    \immediate\write\myauxbrevet{\string\hline}\relax
    \immediate\write\myauxbrevet{Exercice\thecompteurbareme\string & \contenubareme\string\\}\relax
}

\AtEndDocument{
    \immediate\write\myauxbrevet{\string\hline}\relax
    \immediate\write\myauxbrevet{\string\end{tabular}}\relax
    \immediate\closeout\myauxbrevet
}

\newcounter{compteurbareme}
\setcounter{compteurbareme}{0}

\begin{document}

%Paramétrage du document, taille à changer au besoin pour un élève à besoins particuliers
\begin{luacode}
  mesParametres('basique')
  Enonce = true
  Taille = nil
\end{luacode}
\parametrage

%Redéfinition de l'environnement enonce pour avoir une apparence de brevet avec affichage des points
\RenewEnviron{enonce}[1][]{
    \stepcounter{compteurexercice}\large\bfseries Exercice \thecompteurexercice \hfill \contenubareme\par\medskip\mdseries\normalsize\BODY
}
%Footer gauche = repére, footer droite = numérotation
  \ifoot{GENMAT\the\year}
  \ofoot{\mdseries\bfseries\thepage/\pageref{LastPage}}        

%Début de la page de garde
\begin{center}
  \tailletexte{4}\bfseries BREVET BLANC
  
  \tailletexte{4}\bfseries \monthyear
  
  \sautligne
  
\end{center}

\begin{tcolorbox}[
  colframe = black, 
  colback=black!0,
  coltitle=white,
  sharp corners,
  valign=center,
  top=2cm,
  bottom=0.5cm,
  left=0.5cm,
  enhanced,
  before skip=0cm,
]
\begin{center}
  \tailletexte{4}
  \textbf{ MATHÉMATIQUES}
  
  Série Générale
  
  \sautligne
  
  \tailletexte{0}
  
  Durée de l'épreuve : 2h00 \hfill 100 points
\end{center}
\end{tcolorbox}

\vfill

\begin{center}
  \tailletexte{0}
  Dès que le sujet vous est remis, assurez-vous qu’il est complet.

  %\sautligne

  Il comporte \textbf{\pageref{LastPage}} pages numérotées de la page \textbf{1} sur \textbf{\pageref{LastPage}} à la page \textbf{\pageref{LastPage}} sur \textbf{\pageref{LastPage}}.
\end{center}

\sautligne

\begin{center}
  \tailletexte{0}
  L'usage de calculatrice avec mode examen actif est autorisé.

  \sautligne
  
  L'usage de calculatrice sans mémoire << type collège >> est autorisé.
  
  \sautligne

  L'usage du dictionnaire est interdit.
\end{center}

\sautligne

\begin{center}
  \tailletexte{0}
  \tableaubrevet
\end{center}

\vfill

\hfill

\newpage

%Chaque exercice doit être placé dans un environnement enonce, précédé de la commande bareme pour afficher les points de l'exercice

%%%%%%%%%%%%%%%%%%%%%%

\bareme{12}

\begin{enonce}
   Contenu de l'exercice 1
\end{enonce}

%%%%%%%%%%%%%%%%%%%%%%

\bareme{16}

\begin{enonce}
   Contenu de l'exercice 2
\end{enonce}

%%%%%%%%%%%%%%%%%%%%%%

\newpage

\bareme{24}

\begin{enonce}
   Contenu de l'exercice 3 (saut de page pour tester la numérotation automatique)
\end{enonce}





\end{document}