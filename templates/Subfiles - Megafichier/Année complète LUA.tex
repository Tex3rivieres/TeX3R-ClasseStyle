\documentclass{classe-tex3R}
\usepackage{style-tex3R}

\begin{document}

\tableofcontents



\begin{luacode*}

--Cette fonction LUA parcourt le dossier cible. Les dossiers de niveau 1 sont interprétés comme les noms des chapitres. Les fichiers.tex dans les sous-dossiers de chaque chapitre sont récupérés automatiquement et insérés par \subfileinclude
local lfs = require("lfs")

local function parcourirDossier(dossier, dossierRacine, dossierParent)
    if dossierParent == dossierRacine then
        local _, _, nomDossier = string.find(dossier, "([^/]+)$")
        nomDossier = string.gsub(nomDossier or dossier, "^%d*%s*", "")      
        tex.sprint('\\chapitre{' .. nomDossier .. '}')
    end

    for fichier in lfs.dir(dossier) do
        if fichier ~= "." and fichier ~= ".." then
            local chemin = dossier .. "/" .. fichier
            local attributs = lfs.attributes(chemin)

            if attributs.mode == "file" then
                if string.match(fichier, "%.tex$") then
                    tex.sprint('\\subfileinclude{' .. chemin .. '}')
                end
            elseif attributs.mode == "directory" then
                parcourirDossier(chemin, dossierRacine, dossier)
            end
        end
    end
end


local dossierRacine = "./Année complète"

parcourirDossier(dossierRacine, dossierRacine, nil)

\end{luacode*}
\end{document}