\documentclass{classe-tex3R}
\usepackage{style-tex3R}

\begin{luacode}
	Format = 'fiche' --diapo/fiche
	Chapitre = "Grandeurs, aires et périmètres" --nom du chapitre
	Numero = '1' --numéro du chapitre
	Niveau = '5' --niveau de classe
\end{luacode}
\parametrage
%FIN PARAMÉTRAGE PRÉAMBULE

\begin{document}

\begin{luacode}
	mesParametres('TD') --activite/bilan/corrige/cours/DM/DS/flash/interro/TD
\end{luacode}
\parametrage
%FIN PARAMÉTRAGE DOCUMENT

%%%%%%%%%%%%%%%%
%%%% ÉNONCÉ %%%%
%%%%%%%%%%%%%%%%

\source{Vincent Crombez}
\theme{Périmètres}
\competence{Communiquer, Calculer}
\difficulte{1}

\begin{enonce}
	Exprime puis calcule le périmètre de chacune des figures : 

	\begin{tasks}[style=enumerate](2)
		\task Le carré de côté $\np[cm]{6}$
		\task Le rectangle de dimensions $\np[dm]{0.9}$ par $\np[cm]{3.5}$
		\task Le triangle de dimensions $\np[hm]{3}$, $\np[dam]{15}$ et $\np[km]{0.5}$
		\task Le pentagone régulier de côté $\np[mm]{18}$
	\end{tasks}
\end{enonce}


%%%%%%%%%%%%%%%%
%%%% ÉNONCÉ %%%%
%%%%%%%%%%%%%%%%

\source{Vincent Crombez}
\theme{Périmètres}
\competence{Raisonner, Communiquer, Calculer}
\difficulte{1}

\begin{enonce}
	Exprime puis calcule le périmètre de chacune des figures :


\begin{tikzpicture}[line cap=round,line join=round,>=triangle 45,x=1.0cm,y=1.0cm]
\clip(2.20115323570354,-3.5564823909393923) rectangle (20.137505992768205,5.692681237050814);
\draw [shift={(4.,2.)},line width=0.8pt]  plot[domain=0.:3.141592653589793,variable=\t]({1.*1.*cos(\t r)+0.*1.*sin(\t r)},{0.*1.*cos(\t r)+1.*1.*sin(\t r)});
\draw [line width=0.8pt] (3.,2.)-- (3.,-2.);
\draw [line width=0.8pt] (3.1053702185467245,0.) -- (2.8946297814532755,0.);
\draw [line width=0.8pt] (3.,-2.)-- (7.,-2.);
\draw [line width=0.8pt] (5.,-1.8946297814532758) -- (5.,-2.1053702185467236);
\draw [line width=0.8pt] (7.,-2.)-- (7.,0.);
\draw [line width=0.8pt] (6.894629781453276,-1.040977307212616) -- (7.105370218546725,-1.040977307212616);
\draw [line width=0.8pt] (6.894629781453276,-0.9590226927873863) -- (7.105370218546725,-0.9590226927873863);
\draw [line width=0.8pt] (7.,0.)-- (5.,0.);
\draw [line width=0.8pt] (6.040977307212615,-0.10537021854672386) -- (6.040977307212615,0.10537021854672386);
\draw [line width=0.8pt] (5.959022692787386,-0.10537021854672386) -- (5.959022692787386,0.10537021854672386);
\draw [line width=0.8pt] (5.,0.)-- (5.,2.);
\draw [line width=0.8pt] (4.8946297814532755,0.9590226927873863) -- (5.1053702185467245,0.9590226927873863);
\draw [line width=0.8pt] (4.8946297814532755,1.040977307212616) -- (5.1053702185467245,1.040977307212616);
\draw [line width=0.8pt,dash pattern=on 3pt off 3pt] (3.,2.)-- (5.,2.);
\draw [line width=0.8pt] (3.959022692787385,2.1053702185467236) -- (3.959022692787385,1.8946297814532758);
\draw [line width=0.8pt] (4.040977307212615,2.1053702185467236) -- (4.040977307212615,1.8946297814532758);
\draw [shift={(10.5,-2.)},line width=0.8pt]  plot[domain=0.:3.141592653589793,variable=\t]({1.*1.5*cos(\t r)+0.*1.5*sin(\t r)},{0.*1.5*cos(\t r)+1.*1.5*sin(\t r)});
\draw [shift={(14.,-1.)},line width=0.8pt]  plot[domain=1.5707963267948966:4.71238898038469,variable=\t]({1.*1.*cos(\t r)+0.*1.*sin(\t r)},{0.*1.*cos(\t r)+1.*1.*sin(\t r)});
\draw [line width=0.8pt] (12.,-2.)-- (14.,-2.);
\draw [line width=0.8pt] (12.959022692787387,-1.8946297814532758) -- (12.959022692787387,-2.1053702185467236);
\draw [line width=0.8pt] (13.040977307212618,-1.8946297814532758) -- (13.040977307212618,-2.1053702185467236);
\draw [line width=0.8pt] (14.,0.)-- (9.,0.);
\draw [line width=0.8pt] (9.,0.)-- (9.,-2.);
\draw [line width=0.8pt] (9.105370218546724,-0.9590226927873863) -- (8.894629781453276,-0.9590226927873863);
\draw [line width=0.8pt] (9.105370218546724,-1.040977307212616) -- (8.894629781453276,-1.040977307212616);
\draw [line width=0.8pt,dash pattern=on 3pt off 3pt] (9.,-2.)-- (12.,-2.);
\draw [line width=0.8pt] (10.74586384327569,-2.) -- (10.622931921637845,-2.1580553278200854);
\draw [line width=0.8pt] (10.74586384327569,-2.) -- (10.622931921637845,-1.8419446721799138);
\draw [line width=0.8pt,dash pattern=on 3pt off 3pt] (10.5,-2.) -- (10.377068078362154,-2.1580553278200854);
\draw [line width=0.8pt,dash pattern=on 1pt off 1pt] (10.5,-2.) -- (10.377068078362154,-1.8419446721799138);
\draw [line width=0.8pt,dash pattern=on 3pt off 3pt] (14.,-2.)-- (14.,0.);
\draw [line width=0.8pt] (13.894629781453277,-1.040977307212616) -- (14.105370218546726,-1.040977307212616);
\draw [line width=0.8pt] (13.894629781453277,-0.9590226927873863) -- (14.105370218546726,-0.9590226927873863);
\draw [shift={(10.,3.)},line width=0.8pt]  plot[domain=0.7853981633974483:3.9269908169872414,variable=\t]({1.*1.4142135623730951*cos(\t r)+0.*1.4142135623730951*sin(\t r)},{0.*1.4142135623730951*cos(\t r)+1.*1.4142135623730951*sin(\t r)});
\draw [line width=0.8pt] (11.,4.)-- (13.,2.);
\draw [line width=0.8pt] (13.,2.)-- (13.,1.);
\draw [line width=0.8pt] (10.,1.)-- (9.,2.);
\draw [line width=0.8pt] (10.,1.)-- (13.,1.);
\draw [line width=0.8pt] (11.745863843275691,1.) -- (11.622931921637845,0.8419446721799153);
\draw [line width=0.8pt] (11.745863843275691,1.) -- (11.622931921637845,1.158055327820087);
\draw [line width=0.8pt] (11.5,1.) -- (11.377068078362155,0.8419446721799153);
\draw [line width=0.8pt] (11.5,1.) -- (11.377068078362155,1.158055327820087);
\draw [line width=0.8pt,dash pattern=on 3pt off 3pt] (9.,2.)-- (11.,4.);
\draw [shift={(17.,2.)},line width=0.8pt]  plot[domain=-1.5707963267948966:1.5707963267948966,variable=\t]({1.*1.*cos(\t r)+0.*1.*sin(\t r)},{0.*1.*cos(\t r)+1.*1.*sin(\t r)});
\draw [shift={(17.,0.)},line width=0.8pt]  plot[domain=1.5707963267948966:4.71238898038469,variable=\t]({1.*1.*cos(\t r)+0.*1.*sin(\t r)},{0.*1.*cos(\t r)+1.*1.*sin(\t r)});
\draw [line width=0.8pt] (17.,3.)-- (19.,3.);
\draw [line width=0.8pt] (17.959022692787386,3.1053702185467245) -- (17.959022692787386,2.894629781453277);
\draw [line width=0.8pt] (18.040977307212614,3.1053702185467245) -- (18.040977307212614,2.894629781453277);
\draw [line width=0.8pt] (17.,-1.)-- (19.,-1.);
\draw [line width=0.8pt] (17.959022692787386,-0.8946297814532773) -- (17.959022692787386,-1.105370218546725);
\draw [line width=0.8pt] (18.040977307212614,-0.8946297814532773) -- (18.040977307212614,-1.105370218546725);
\draw [shift={(19.,2.)},line width=0.8pt]  plot[domain=-1.5707963267948966:1.5707963267948966,variable=\t]({1.*1.*cos(\t r)+0.*1.*sin(\t r)},{0.*1.*cos(\t r)+1.*1.*sin(\t r)});
\draw [shift={(19.,0.)},line width=0.8pt]  plot[domain=1.5707963267948966:4.71238898038469,variable=\t]({1.*1.*cos(\t r)+0.*1.*sin(\t r)},{0.*1.*cos(\t r)+1.*1.*sin(\t r)});
\draw [line width=0.8pt,dash pattern=on 3pt off 3pt] (17.,1.)-- (17.,-1.);
\draw [line width=0.8pt] (17.105370218546724,0.04097730721261484) -- (16.894629781453276,0.04097730721261484);
\draw [line width=0.8pt] (17.105370218546724,-0.04097730721261484) -- (16.894629781453276,-0.04097730721261484);
\draw [line width=0.8pt,dash pattern=on 3pt off 3pt] (19.,3.)-- (19.,1.);
\draw [line width=0.8pt] (19.105370218546728,2.0409773072126156) -- (18.894629781453276,2.0409773072126156);
\draw [line width=0.8pt] (19.105370218546728,1.9590226927873862) -- (18.894629781453276,1.9590226927873862);
\draw [line width=0.8pt,dash pattern=on 3pt off 3pt] (17.,3.)-- (17.,1.);
\draw [line width=0.8pt] (17.105370218546724,2.0409773072126156) -- (16.894629781453276,2.0409773072126156);
\draw [line width=0.8pt] (17.105370218546724,1.9590226927873862) -- (16.894629781453276,1.9590226927873862);
\draw [line width=0.8pt,dash pattern=on 3pt off 3pt] (19.,1.)-- (19.,-1.);
\draw [line width=0.8pt] (19.105370218546728,0.04097730721261484) -- (18.894629781453276,0.04097730721261484);
\draw [line width=0.8pt] (19.105370218546728,-0.04097730721261484) -- (18.894629781453276,-0.04097730721261484);
\draw (4.402220023124008,-2.0578837271637638) node[anchor=north west] {$\np[cm]{3,2}$};
\draw (5.192012979740793,1.488612932275307) node[anchor=north west] {$\np[cm]{1,8}$};
\draw (9.75805578343219,-2.08870976621694) node[anchor=north west] {$\np[cm]{2,3}$};
\draw (10.443445886469549,0.5893136003874928) node[anchor=north west] {$\np[cm]{4,1}$};
\draw (9.038509639179887,3.9130953238172554) node[anchor=north west] {$\np[cm]{5,2}$};
\draw (8.059210307292068,1.646184577611743) node[anchor=north west] {$\np[cm]{1,4}$};
\draw (11.988875758488176,3.6428485114527726) node[anchor=north west] {$\np[cm]{2,7}$};
\draw (13.053318381048869,2.0335094105836965) node[anchor=north west] {$\np[cm]{0,9}$};
\end{tikzpicture}
\end{enonce}

%%%%%%%%%%%%%%%%
%%%% ÉNONCÉ %%%%
%%%%%%%%%%%%%%%%

\source{Vincent Crombez}
\theme{Conversions}
\competence{Calculer, Raisonner}
\difficulte{1}

\begin{enonce}
	Convertis chacune des longueurs suivantes :

	\begin{tasks}[style=enumerate](4)
		\task $\np[hm]{2} = \dots\,\text{m}$
		\task $\np[dam]{54} = \dots\,\text{cm}$
		\task $\np[mm]{329} = \dots\,\text{dam}$
		\task $\np[m]{45.36} = \dots\,\text{dm}$
		\task $\np[cm]{23} = \dots\,\text{km}$
		\task $\np[m]{48.52} = \dots\,\text{km}$
		\task $\np[cm]{25.2} = \dots\,\text{hm}$
		\task $\np[dam]{13} = \dots\,\text{hm}$
	\end{tasks}
\end{enonce}

%%%%%%%%%%%%%%%%
%%%% ÉNONCÉ %%%%
%%%%%%%%%%%%%%%%

\source{Vincent Crombez}
\theme{Aire}
\competence{Calculer, Communiquer}
\difficulte{1}

\begin{enonce}
	Calcule l'aire de chacune des figures :

	\begin{enumerate}[leftmargin=*]
			\item Un rectangle de largeur $\np[cm]{3}$ et de longueur $\np[cm]{7}$.
			\item Un carré de côté $\np[cm]{8}$.
			\item Un triangle rectangle dont les côtés de l'angle droit mesurent $\np[cm]{5}$   et $\np[cm]{6}$.
			\item Un triangle de hauteur $\np[cm]{4}$ et de base $\np[cm]{6.5}$.
			\item Un disque de rayon $\np[cm]{10}$.
			\item Un parallélogramme dont les mesures de deux côtés sont $\np[cm]{8}$ et $\np[cm]{4.5}$
	\end{enumerate}
\end{enonce}

\newpage

%%%%%%%%%%%%%%%%
%%%% ÉNONCÉ %%%%
%%%%%%%%%%%%%%%%

\source{Vincent Crombez}
\theme{Aires}
\competence{Raisonner, Communiquer, Calculer}
\difficulte{2}

\begin{enonce}
	Calcule l'aire des figures suivantes :

\saut{ligne} %#1=fiche/diapo

\adjustbox{valign=t}{\begin{minipage}{0.48\linewidth}%
		\begin{tikzpicture}
			\tkzDefPoints{0/4/A,0/0/B,3/0/C,3/4/D,5/0/E,0/-1/Y,5/-1/Z}
			\tkzLabelPoints[above](A,D)
			\tkzLabelPoints[below](B,C,E)
			\tkzDrawSegments(A,B B,C C,E E,D D,A)
			\tkzDrawSegments[style=dashed](D,C)
			\tkzMarkRightAngle(D,A,B)
			\tkzMarkRightAngle(A,B,C)
			\tkzMarkRightAngle(B,C,D)
			\tkzMarkRightAngle(C,D,A)
			\tkzDrawSegment[{Latex[scale=1.5]}-{Latex[scale=1.5]}](Y,Z)
			\tkzLabelSegment[pos=0.5,below](Y,Z){$\np[cm]{7.5}$}
			\tkzLabelSegment[pos=0.5,left](A,B){$\np[cm]{5.6}$}
			\tkzLabelSegment[pos=0.5,above](A,D){$\np[cm]{3.2}$}
		\end{tikzpicture}
\end{minipage}}\hfill%
\adjustbox{valign=t}{\begin{minipage}{0.48\linewidth}%
		\begin{tikzpicture}
		\tkzDefPoints{0/2/A,4/2/B,4/0/C,0/0/D,0/1/E,4/1/F,5.5/0/Y,5.5/2/Z}
		\tkzLabelPoints[above](A,B)
		\tkzLabelPoints[below](C,D)
		\tkzDrawSegments(A,B C,D)
		\tkzDrawSegments[style=dashed](B,C D,A)
		\tkzDrawArc(E,A)(D)
		\tkzDrawArc(F,C)(B)
		\tkzMarkRightAngle(D,A,B)
		\tkzMarkRightAngle(A,B,C)
		\tkzMarkRightAngle(B,C,D)
		\tkzMarkRightAngle(C,D,A)
		\tkzLabelSegment[pos=0.5,below](D,C){$\np[cm]{8.4}$}
		\tkzDrawSegment[{Latex[scale=1.5]}-{Latex[scale=1.5]}](Y,Z)
		\tkzLabelSegment[pos=0.5,right](Y,Z){$\np[cm]{2.1}$}
	\end{tikzpicture}

\vspace{0.2cm}

\begin{tikzpicture}
	\tkzDefPoints{0/0/A,4/0/B,1/1/D,5/1/C,5/2.41/E,6/0/Y,6/1/Z}
	\tkzDrawSegments(A,B B,C C,E E,D D,A)
	\tkzDrawSegments[style=dashed](D,Z B,Y)
	\tkzLabelPoints[above](E)
	\tkzLabelPoints[below](A,B)
	\tkzLabelPoints[right, below](C)
	\tkzLabelPoints[left](D)
	\tkzMarkSegments[pos=0.5,mark=s||](A,B D,C)
	\tkzMarkSegments[pos=0.5,mark=s|](A,D B,C C,E)
	\tkzMarkRightAngle(D,C,E)
	\tkzLabelSegment[pos=0.5,below=0.2](A,B){$\np[cm]{4.7}$}
		\tkzLabelSegment[pos=0.5,right](E,C){$\np[cm]{1.2}$}
	\tkzDrawSegment[{Latex[scale=1.5]}-{Latex[scale=1.5]}](Y,Z)
	\tkzLabelSegment[pos=0.5,right](Y,Z){$\np[cm]{0.8}$}
\end{tikzpicture}
\end{minipage}}%

\end{enonce}

%%%%%%%%%%%%%%%%
%%%% ÉNONCÉ %%%%
%%%%%%%%%%%%%%%%

\source{Vincent Crombez}
\theme{Conversions}
\competence{Calculer, Raisonner}
\difficulte{1}

\begin{enonce}
	Convertis chacune des aires suivantes :

	\begin{tasks}[style=enumerate](4)
		\task $\np[cm^2]{372} = \dots\,\text{dam}^2$
		\task $\np[m^2]{4807} = \dots\,\text{hm}^2$
		\task $\np[km^2]{0.005} = \dots\,\text{m}^2$
		\task $\np[dam^2]{414} = \dots\,\text{cm}^2$	
		\task $\np[hm^2]{8.36} = \dots\,\text{mm}^2$
		\task $\np[mm^2]{28} = \dots\,\text{cm}^2$
		\task $\np[cm^2]{405.2} = \dots\,\text{m}^2$
		\task $\np[hm^2]{0.52} = \dots\,\text{km}^2$
	\end{tasks}
\end{enonce}

%%%%%%%%%%%%%%%%
%%%% ÉNONCÉ %%%%
%%%%%%%%%%%%%%%%

\source{Sesamath}
\theme{Aires}
\competence{Raisonner, Communiquer}
\difficulte{3}

\begin{enonce}
Un pâtissier doit confectionner une tarte recouverte de glaçage. Il sait qu'avec $\np[g]{100}$ de sucre glace, il fabrique du glaçage pour une surface de $\np[dm^2]{5}$. Sachant qu'il dispose de moules à tarte circulaires de diamètres $\np[cm]{22}$, $\np[cm]{26}$ ou $\np[cm]{28}$, quel moule devra-t-il utiliser pour $\np[g]{100}$ de sucre ?
\end{enonce}



%%%%%%%%%%%%%%%%
%%%% ÉNONCÉ %%%%
%%%%%%%%%%%%%%%%

\source{Sesamath}
\theme{Aires et périmètres}
\competence{Raisonner, Représenter}
\difficulte{3}


\begin{enonce}
	Un peintre en bâtiment fait l'expérience suivante : il imbibe entièrement son rouleau de peinture, il le pose sur le mur, le fait rouler en lui faisant faire seulement un tour complet, puis le retire du mur.

	\begin{enumerate}
		\item Quelle va être la forme de la tache de peinture ainsi réalisée ?
		\item Le rouleau est large de $\np[cm]{25}$ et d'un diamètre de $\np[cm]{8}$. Quelle surface du mur sera alors recouverte de peinture ?
		\item Combien de fois au minimum devra-t-il réaliser ce geste pour peindre un mur long de $\np[m]{6}$ et haut de $\np[m]{2.5}$ ?
	\end{enumerate}
\end{enonce}

%%%%%%%%%%%%%%%%
%%%% ÉNONCÉ %%%%
%%%%%%%%%%%%%%%%

\source{Sesamath}
\theme{Aires}
\competence{Chercher, Communiquer, Représenter}
\difficulte{3}

\begin{enonce}
	Construis un parallélogramme qui a un côté de $\np[cm]{6}$ de longueur, un périmètre de $\np[cm]{20}$ et une aire de $\np[cm^2]{18}$. Justifie ta construction en indiquant tes calculs.
\end{enonce}

\end{document}