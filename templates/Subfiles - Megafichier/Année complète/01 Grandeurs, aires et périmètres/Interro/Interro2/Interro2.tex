\documentclass{classe-tex3R}
\usepackage{style-tex3R}

\begin{document}

\definirchapitre{1}{Grandeurs, aires et périmètres}
\definirniveau{5}

\begin{luacode}
  mesParametres('interro')
  
\end{luacode}
\parametrage

\source{Vincent Crombez}
\theme{Aire}
\competence{Communiquer}
\difficulte{1}

\begin{enonce}
	En utilisant certains mots de la boîte suivante, exprime les formules de surface de chacune des figures :

\medskip

\fbox{\begin{minipage}{\linewidth}
  base \hfill longueur  \hfill hauteur \hfill rayon \hfill profondeur \hfill médiane \hfill côté \hfill largeur \hfill diamètre 
\end{minipage}}

\medskip

\begin{enumerate}
\setlength{\itemsep}{0.5cm}
  \item Le rectangle : \dotfill
  \item Le carré : \dotfill
  \item Le disque : \dotfill
  \item Le parallélogramme : \dotfill
  \item Le triangle : \dotfill
\end{enumerate}
\end{enonce}

\vspace{6.5cm}

\setcounter{compteurinterro}{0}

\titreactif

\sautdeligne

\setcounter{compteurexercice}{0}

\source{Vincent Crombez}
\theme{Aire}
\competence{Communiquer}
\difficulte{1}

\begin{enonce}
	En utilisant certains mots de la boîte suivante, exprime les formules de surface de chacune des figures :

\medskip

\fbox{\begin{minipage}{\linewidth}
  base \hfill longueur  \hfill hauteur \hfill rayon \hfill profondeur \hfill médiane \hfill côté \hfill largeur \hfill diamètre 
\end{minipage}}

\medskip

\begin{enumerate}
\setlength{\itemsep}{0.5cm}
  \item Le carré : \dotfill
  \item Le disque : \dotfill
  \item Le rectangle : \dotfill
  \item Le triangle : \dotfill
  \item Le parallélogramme : \dotfill
\end{enumerate}
\end{enonce}


\newpage

%%%%%%%%%%%%%%%%
%%%% ÉNONCÉ %%%%
%%%%%%%%%%%%%%%%


\setcounter{compteurexercice}{1}

\source{Vincent Crombez}
\theme{Aire}
\competence{Modéliser, Communiquer}
\difficulte{1}

\begin{enonce}
Pour chaque figure, représente la à main levée, puis exprime et calcule son aire :

  	\begin{enumerate}[leftmargin=*]
			\item Un rectangle dont les dimensions sont $\np[cm]{4}$ par $\np[cm]{6}$.
			

\dotfill

\medskip

\dotfill

\medskip

\dotfill

\medskip

\dotfill

			
			\item Un triangle de hauteur $\np[cm]{6}$ et de base $\np[cm]{4.5}$.
			      

\dotfill

\medskip

\dotfill

\medskip

\dotfill

\medskip

\dotfill


			\item Un disque de diamètre $\np[cm]{12}$.
			

\dotfill

\medskip

\dotfill

\medskip

\dotfill

\medskip

\dotfill

	\end{enumerate}
\end{enonce}

%%%%%%%%%%%%%%%%
%%%% ÉNONCÉ %%%%
%%%%%%%%%%%%%%%%

\vfill

\setcounter{compteurexercice}{1}

\source{Vincent Crombez}
\theme{Aire}
\competence{Modéliser, Communiquer}
\difficulte{1}

\begin{enonce}
Pour chaque figure, représente la à main levée, puis exprime et calcule son aire :

  	\begin{enumerate}[leftmargin=*]
			\item Un rectangle dont les dimensions sont $\np[cm]{5}$ par $\np[cm]{9}$.
			

\dotfill

\medskip

\dotfill

\medskip

\dotfill

\medskip

\dotfill

			
			\item Un triangle de hauteur $\np[cm]{6}$ et de base $\np[cm]{3.5}$.
			      

\dotfill

\medskip

\dotfill

\medskip

\dotfill

\medskip

\dotfill


			\item Un disque de diamètre $\np[cm]{10}$.
			

\dotfill

\medskip

\dotfill

\medskip

\dotfill

\medskip

\dotfill


	\end{enumerate}


\end{enonce}

\end{document}