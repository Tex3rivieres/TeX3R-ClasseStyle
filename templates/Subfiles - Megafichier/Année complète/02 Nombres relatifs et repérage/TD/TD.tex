\documentclass[../Chapitre2.tex]{subfiles}

\begin{document}

\definirchapitre{2}{Nombres relatifs et repérage}
\definirniveau{5}

\begin{luacode}
  mesParametres('TD')
\end{luacode}
\parametrage

%%%%%%%%%%%%%%%%
%%%% ÉNONCÉ %%%%
%%%%%%%%%%%%%%%%

\source{Vincent Crombez}
\theme{Nombres relatifs}
\competence{Raisonner}
\difficulte{1}

\begin{enonce}
  Classe les nombres suivants dans l'ordre croissant : \hfill $-7$ \hfill $+8$ \hfill $-3$ \hfill $15$ \hfill $-11$
\end{enonce}

\source{Vincent Crombez}
\theme{Nombres relatifs}
\competence{Raisonner}
\difficulte{2}

\begin{enonce}
  Classe les nombres suivants dans l'ordre croissant : \hfill $-6$ \hfill $-\np{5.5}$ \hfill $3$ \hfill $\np{4.5}$ \hfill $-\np{6.2}$
\end{enonce}

%%%%%%%%%%%%%%%%
%%%% ÉNONCÉ %%%%
%%%%%%%%%%%%%%%%

\source{Vincent Crombez}
\theme{Nombres relatifs}
\competence{Raisonner}
\difficulte{1}

\begin{enonce}
  Compare les nombres suivants entre eux :

  \begin{tasks}[style=itemize](4)
    \task $\np{4.3}$ et $\np{3.4}$
    \task $-\np{6.02}$ et $-\np{2.06}$
    \task $-\np{25.04}$ et $-\np{25.4}$
    \task $-\np{8.25}$ et $-\np{8.26}$
  \end{tasks}
\end{enonce}

%%%%%%%%%%%%%%%%
%%%% ÉNONCÉ %%%%
%%%%%%%%%%%%%%%%

\source{Vincent Crombez}
\theme{Nombres relatifs}
\competence{Représenter}
\difficulte{2}

\begin{enonce}
Pour chacun des axes ci-dessous, indique l'abscisse des points demandés :

\bigskip

  \Reperage[AffichageNom,Unitex=0.75]{-5/,2/A,-3/B,5/}\hfill\Reperage[AffichageNom,Pasx=2,Unitex=0.75]{-10/,6/C,-3/D,10/}

\vspace{1cm}

\Reperage[AffichageNom,Unitex=0.75]{-8/,-6/E,-1/F,2/}\hfill\Reperage[AffichageNom,Pasx=4,Unitex=0.75]{-29/,-10/G,-17/H,12/}

\end{enonce}

%%%%%%%%%%%%%%%%
%%%% ÉNONCÉ %%%%
%%%%%%%%%%%%%%%%

\source{Vincent Crombez}
\theme{Nombres relatifs}
\competence{Raisonner, Représenter}
\difficulte{2}

\begin{enonce}
  Construis un axe en prenant un centimètre pour 100 ans, et place le plus précisément possible les dates de naissance des mathématiciens grecs suivants :

\begin{tasks}[style=itemize](3)
  \task Thalès : $-620$
  \task Anaximandre : $-610$
  \task Pythagore : $-580$
  \task Eratosthène : $-280$
  \task Euclide  : $-300$
  \task Zénon : $-490$
\end{tasks}

\end{enonce}

%%%%%%%%%%%%%%%%
%%%% ÉNONCÉ %%%%
%%%%%%%%%%%%%%%%

\source{Vincent Crombez}
\theme{Nombres relatifs}
\competence{Représenter}
\difficulte{1}

\begin{enonce}

\begin{enumerate}
  \item Donne les coordonnées de chacun des points dans le repère ci-dessous.
  \item Place les points suivants : \hfill $H(7\,;\,2)$ \hfill  $I(-3\,;\,3)$ \hfill  $J(-8\,;\,0)$ \hfill  $K(2\,;\,-3)$ \hfill
\end{enumerate}

\bigskip

  \scalebox{0.8}{\Reperage[Plan,AffichageNom]{-8/-2/,8/2/,-5/-2/A,-7/3/B,1/3/C,4/-1/D,8/-3/E,-2/-3/F,3/0/G}}
\end{enonce}

\end{document}