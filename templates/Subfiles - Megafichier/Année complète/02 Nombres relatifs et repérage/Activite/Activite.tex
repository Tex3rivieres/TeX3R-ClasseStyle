\documentclass[../Chapitre2.tex]{subfiles}

\begin{document}

\definirchapitre{2}{Nombres relatifs et repérage}
\definirniveau{5}

\begin{luacode}
  mesParametres('activite')
\end{luacode}
\parametrage

%%%%%%%%%%%%%%%%
%%%% ÉNONCÉ %%%%
%%%%%%%%%%%%%%%%

\source{Vincent Crombez}
\theme{Nombres relatifs}
\competence{Raisonner}
\difficulte{1}

\begin{enonce}

Pour chacun des thermomètres suivants, donne leur température :

\bigskip

\Reperage[%
Thermometre,%
AffichageGrad,%
Unitex=0.5,
Mercure]{%
7/A,-4/,8/}
%
\hfill
%
\Reperage[%
Thermometre,%
AffichageGrad,%
Unitex=0.5,
Mercure]{%
-2/A,-6/,6/}
%
\hfill
%
\Reperage[%
Thermometre,%
Unitex=0.5,
Mercure]{%
5/A,-3/,9/}
%
\hfill
%
\Reperage[%
Thermometre,%
Unitex=0.5,
Mercure]{%
-5/A,-8/,4/}
%
\hfill
%
\Reperage[%
Thermometre,%
Pasx=4,
ValeurUnitex=1,
Mercure]{%
9/A,-8/,14/}
%
\hfill
%
\Reperage[%
Thermometre,%
Pasx=4,
ValeurUnitex=1,
Mercure]{%
-6/A,-12/,10/}

\end{enonce}

%%%%%%%%%%%%%%%%
%%%% ÉNONCÉ %%%%
%%%%%%%%%%%%%%%%

\source{Vincent Crombez}
\theme{Nombres relatifs}
\competence{Représenter, Raisonner}
\difficulte{1}

\begin{enonce}
  Représente la température demandée sur chacun des thermomètres :

\bigskip

$\np[^\circ C]{+6}$ \hfill $\np[^\circ C]{-4}$ \hfill $\np[^\circ C]{7}$ \hfill $\np[^\circ C]{-5}$ \hfill $\np[^\circ C]{-1.75}$ \hfill $\np[^\circ C]{-2.25}$

\bigskip

\Reperage[%
Thermometre,%
Unitex=0.5%,
%Mercure
]{%
6/A,-4/,8/}
%
\hfill
%
\Reperage[%
Thermometre,%
Unitex=0.5%,
%Mercure
]{%
-4/A,-6/,6/}
%
\hfill
%
\Reperage[%
Thermometre,%
Unitex=0.5%,
%Mercure
]{%
7/A,-3/,9/}
%
\hfill
%
\Reperage[%
Thermometre,%
Unitex=0.5%,
%Mercure
]{%
-5/A,-8/,4/}
%
\hfill
%
\Reperage[%
Thermometre,%
Pasx=4,
ValeurUnitex=1%,
%Mercure
]{%
-7/A,-8/,14/}
%
\hfill
%
\Reperage[%
Thermometre,%
Pasx=4,
ValeurUnitex=1%,
%Mercure
]{%
-9/A,-12/,10/}
\end{enonce}

%%%%%%%%%%%%%%%%
%%%% ÉNONCÉ %%%%
%%%%%%%%%%%%%%%%

\source{Vincent Crombez}
\theme{Vincent Crombez}
\competence{Raisonner}
\difficulte{1}

\begin{enonce}
  Classe toutes ces températures de la plus froide à la plus chaude :

\bigskip

  $\np[^\circ C]{-10}$ \hfill $\np[^\circ C]{+15}$ \hfill $\np[^\circ C]{-12}$ \hfill $\np[^\circ C]{5}$ \hfill $\np[^\circ C]{0}$ \hfill $\np[^\circ C]{-14.5}$ \hfill $\np[^\circ C]{-14.75}$  
\end{enonce}

%%%%%%%%%%%%%%%%
%%%% ÉNONCÉ %%%%
%%%%%%%%%%%%%%%%

\source{Vincent Crombez}
\theme{Nombres relatifs}
\competence{Raisonner, Calculer}
\difficulte{2}

\begin{enonce}
  Le degré Fahrenheit ($^\circ \text{F}$) est une unité de température utilisée aux États-Unis. Pour convertir des degrés Fahrenheit en degré Celsius, il faut soustraire $32$ à la valeur en Fahrenheit, puis diviser par $\np{1.8}$.
  
  Classe ces températures de la plus froide à la plus chaude :

  \bigskip

  $\np[^\circ F]{32}$ \hfill $\np[^\circ C]{-7}$ \hfill $\np[^\circ F]{41}$ \hfill $\np[^\circ C]{-11}$ \hfill $\np[^\circ C]{+3}$ \hfill $\np[^\circ F]{35.6}$
\end{enonce}

\end{document}

\end{document}