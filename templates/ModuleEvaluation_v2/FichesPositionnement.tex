\documentclass{classe-tex3R}
\usepackage{style-tex3R}

\AtBeginDocument{\npproductsign{\times}\npthousandsep{\,}\npthousandthpartsep{\,}}
\newcommand*\circled[1]{\tikz[baseline=(char.base)]{
            \node[shape=circle,draw,inner sep=2pt] (char) {#1};}}
\setlength\parindent{0pt}

\begin{luacode}
  Format = 'fiche' --diapo/fiche
  Chapitre = '' --nom du chapitre
  Numero = '' --numéro du chapitre
  Niveau = '' --niveau de classe
\end{luacode}
\parametrage
%FIN PARAMÉTRAGE PRÉAMBULE

\begin{document}

\begin{luacode}
  Type = 'fiche' --activite/bilan/corrige/cours/DM/DS/flash/interro/TD
  Visible = false
\end{luacode}
\parametrage
%FIN PARAMÉTRAGE DOCUMENT


%%%  Fonction luacode pour traitement csv et publipostage
\begin{luacode*}
----------------------------   Variables  Configuration -------------------------------
afficheFichePosition=true
afficheFicheRemediation  = false
EnteteLigneBareme = 'Point Max'
NomFichierCSV = 'TableauNotes.csv' -- UTF8, ";"
Niveau = '3ème'
TitreSommaire = 'DS4'
TitreFichePositionnement = 'DS4 - Positionnement'
TitreFicheRemediation = 'DS4 - Remédiation'
PalierCompNiv2=0.3
PalierCompNiv3=0.6
PalierCompNiv4=0.9
NoteEnPourcentage=true

LettreAbs='A'
----------------------------   Variables  Configuration -------------------------------



 -- Fonction affichage tableau pour debug
function AfficheTab(csvtable)
  for i = 1, #csvtable do
    for j = 1, #csvtable[1] do 
      if csvtable[i][j] ~= nil then  -- Vérification de nil
        tex.sprint(csvtable[i][j] .. " ; ")  -- Ajout d'espace après ;
      end
    end
    tex.sprint("\\\\ ")  -- Ajout d'espace après \\
  end
end


-- IMPORT fichier CSV dans csvtable
function csvtotable(filename)
    local fichier = io.open(filename, "r")
    if not fichier then
        error("Impossible d'ouvrir le fichier : " .. filename)
    end

    local csvtable = {}
    local numeroLigne = 0

    for ligne in fichier:lines() do
        numeroLigne = numeroLigne + 1
        local lignecsvtable = {}

        if numeroLigne == 1 then
            -- Traitement pour la première ligne
            for value in ligne:gmatch("[^;]+") do
                value = value:gsub(",", ".")  -- Remplacer les virgules par des points
                table.insert(lignecsvtable, value)
            end
        else
            -- Traitement pour les autres lignes
            for value in ligne:gmatch("([^;]*);?") do
                value = value:gsub(",", ".")  -- Remplacer les virgules par des points
                table.insert(lignecsvtable, value)
            end
        end

        table.insert(csvtable, lignecsvtable)
    end

    fichier:close()
    return csvtable
end




  function chercheligne(csvtable,string)
    local numligne = 1

    while numligne <= #csvtable and csvtable[numligne][1] ~= string do
      numligne = numligne + 1
    end

    if numligne <= #csvtable then
      return numligne
    else
      return nil  -- Renvoie nil si la chaîne n'est pas trouvée
    end
  end


  function ExtractComp(csvtable)
    local RefPtMax=chercheligne(csvtable,EnteteLigneBareme)
    local TabComp={}
    local LigneTabComp={}
    local ContentNewFile=''
    

    --- Entête du tableau avec nom prénom;compétences évaluées et création des exercices de remédiation associés
    table.insert(LigneTabComp,"Nom Prénom")
    for i=4,RefPtMax-1 do
    local ScoreComp=0
        for j=2,#csvtable[1] do
          if tonumber(csvtable[i][j]) ~= nil then
              ScoreComp = ScoreComp + csvtable[i][j]
          end
        end
        if ScoreComp ~= 0 then
            table.insert(LigneTabComp,csvtable[i][1])
            -- creation exercices remediations en fonction des competences evaluées
            nomFichier1 = "./Remediation/" .. csvtable[i][1] .. "-niv1.tex"
            nomFichier2 = "./Remediation/" .. csvtable[i][1] .. "-niv2.tex"

            fichier = io.open(nomFichier1, "r")
            if fichier then
                fichier:close()
            else
                fichier = io.open(nomFichier1, "w")
                ContentNewFile = "\\documentclass[../FichesPositionnement.tex]{subfiles}\n"
                .."\\begin{document}\n"
                .."\\begin{tcolorbox}[breakable=false, boxrule=0pt, colback=white, colframe=white]\n"
                .."\\textbf{Exercices de remediation " ..  csvtable[i][1] .. " niveau 1}\n"
                .."\n"
                .."% Enoncé ici %\n"
                .."\n"
                .."\\end{tcolorbox}\n"
                .."\n"
                .."\n"
                .."%%%%  CORRECTION  %%%%%%%\n"
                .."\\begin{visible}[true]\n"
                .."\\begin{tcolorbox}[breakable=false, boxrule=0pt, colback=white, colframe=white]\n"
                .."\\textbf{Correction}\n"
                .."\n"
                .." % Correction ici %\n"
                .."\n"
                .."\\end{tcolorbox}\n"
                .."\\end{visible}\n"
                .."\\end{document}"
                fichier:write(ContentNewFile)
            end

            fichier = io.open(nomFichier2, "r")
            if fichier then
                fichier:close()
            else
                fichier = io.open(nomFichier2, "w")
                ContentNewFile = "\\documentclass[../FichesPositionnement.tex]{subfiles}\n"
                .. "\\begin{document}\n"
                .."\\begin{tcolorbox}[breakable=false, boxrule=0pt, colback=white, colframe=white]\n"
                .."\\textbf{Exercices de remediation " ..  csvtable[i][1] .. " niveau 2}\n"
                .."\n"
                .."% Enoncé ici %\n"
                .."\n"
                .."\\end{tcolorbox}\n"
                .."\n"
                .."\n"
                .."%%%%  CORRECTION  %%%%%%%\n"
                .." \\begin{visible}[true]\n"
                .." \\begin{tcolorbox}[breakable=false, boxrule=0pt, colback=white, colframe=white]\n"
                .. "\\textbf{Correction}\n"
                .."\n"
                .."% Correction ici %\n"
                .."\n"
                .." \\end{tcolorbox}\n"
                .." \\end{visible}\n"
                .. "\\end{document}"
                fichier:write(ContentNewFile)
            end

        end
    end
    table.insert(TabComp,LigneTabComp)

    --- Ajout lignes élèves avec ratios obtenu par compétence
    for E=RefPtMax+1,#csvtable do
        LigneTabComp={csvtable[E][1]}
    
        for C=4,RefPtMax-1 do
            local ScoreCompMax=0
            local ScoreEleveComp=0
            local RatioComp=0
            local NiveauComp=0
          

            for j=2,#csvtable[1] do
              if tonumber(csvtable[C][j]) ~= nil and tonumber(csvtable[RefPtMax][j])~=nil then
                -- if type(csvtable[C][j]) == "number" and type(csvtable[RefPtMax][j])== "number" then
                    ScoreCompMax = ScoreCompMax + (csvtable[C][j]*csvtable[RefPtMax][j])
                -- end
              end
              if tonumber(csvtable[C][j]) ~= nil then
                if tonumber(csvtable[E][j])~=nil then
                  ScoreEleveComp = ScoreEleveComp + (csvtable[C][j]*csvtable[E][j])
                else ScoreEleveComp = -1
                end
              end
            end
                  
          
          if ScoreCompMax ~= 0  then
            if tonumber(ScoreEleveComp) ~= nil then
            -- if type(ScoreEleveComp) == "number" then
                RatioComp = math.floor(ScoreEleveComp/ScoreCompMax *100 +0.5)/100
                if RatioComp > PalierCompNiv4 then
                  NiveauComp = 4
                elseif RatioComp > PalierCompNiv3 then
                  NiveauComp = 3
                elseif RatioComp > PalierCompNiv2 then
                  NiveauComp = 2
                elseif  RatioComp >= 0 then
                NiveauComp = 1
                else
                NiveauComp = LettreAbs
                end
            else 
            NiveauComp = LettreAbs
            end
                table.insert(LigneTabComp,NiveauComp)
            -- end
           end
        end
        table.insert(TabComp,LigneTabComp)
    end
   
    return TabComp
  end


  function ExtractNotesExercices(csvtable)
    local RefPtMax=chercheligne(csvtable,EnteteLigneBareme)
    local TabNotesExercices={}
    local LigneTabNotesExercices={}
    local NumEx=1
    local PtsEleveEx=0
    local PtsMaxEx=0
    local TotPts=0


    -- Entête
    table.insert(LigneTabNotesExercices,"Nom Pénom")

    for j=2,#csvtable[1] do

      if csvtable[1][j] ~= tostring(NumEx) then
          table.insert(LigneTabNotesExercices,NumEx)
          NumEx = NumEx + 1
      end
    end
    table.insert(LigneTabNotesExercices,NumEx)
    table.insert(LigneTabNotesExercices,"Total")
    table.insert(TabNotesExercices,LigneTabNotesExercices)


    -- Bareme
    LigneTabNotesExercices={}
    PtsMaxEx=0
    NumEx=1
    TotPts=0
    table.insert(LigneTabNotesExercices,EnteteLigneBareme)
    for j=2,#csvtable[1] do
        if csvtable[1][j] ~= tostring(NumEx) then
          table.insert(LigneTabNotesExercices,PtsMaxEx)--           
          NumEx = NumEx + 1
          PtsMaxEx=0
        end
        
        PtsMaxEx = PtsMaxEx + csvtable[RefPtMax][j]
      end
      if PtsMaxEx%1==0 then PtsMaxEx=math.floor(PtsMaxEx) end


      table.insert(LigneTabNotesExercices,PtsMaxEx)
      for i=2,#LigneTabNotesExercices do
          TotPts = TotPts + LigneTabNotesExercices[i]
        end
        if TotPts%1==0 then TotPts=math.floor(TotPts) end
      table.insert(LigneTabNotesExercices,TotPts)
      table.insert(TabNotesExercices,LigneTabNotesExercices)

    --- Ajout lignes élèves avec total points par exercices

    for i=RefPtMax+1,#csvtable do
    TotPts=0
    NumEx=1
    PtsEleveEx=0
    
    LigneTabNotesExercices={}
    table.insert(LigneTabNotesExercices,csvtable[i][1])
      
      for j=2,#csvtable[1] do
        if csvtable[1][j] ~= tostring(NumEx) then
          table.insert(LigneTabNotesExercices,PtsEleveEx)--           
          NumEx = NumEx + 1
          PtsEleveEx=0
        end
        if tonumber(csvtable[i][j])~=nil then
          PtsEleveEx = PtsEleveEx + csvtable[i][j]
        else PtsEleveEx= LettreAbs
        end 
      end
      table.insert(LigneTabNotesExercices,PtsEleveEx)
        for i=2,#LigneTabNotesExercices do
          if tonumber(LigneTabNotesExercices[i])~=nil and tonumber(TotPts)~=nil then        
            TotPts = TotPts + LigneTabNotesExercices[i]
                else TotPts = LettreAbs
          end
        end
        if tonumber(TotPts)~=nil then
        if TotPts%1==0 then TotPts=math.floor(TotPts) end

                end



      table.insert(LigneTabNotesExercices,TotPts)
      table.insert(TabNotesExercices,LigneTabNotesExercices)

    end
    return TabNotesExercices
  end


---------------------------------------------------------
-----   CREATION FICHE SYNTHESE ET REMEDIATION ----------
---------------------------------------------------------

function CreateFichesSynthese(csvtable,TabComp,TabNotesExercices)
  local RefPtMax=chercheligne(csvtable,EnteteLigneBareme)
  local FicheEleveTex=''
  local NomCompetence=''
  local NivComp=0
  local NoteAffiche=0
  local FicheRemediation=''
  
  -- Entête
  for i=RefPtMax+1,#csvtable do
    FicheEleveTex = "\\phantomsection\\addcontentsline{toc}{subsection}{" .. csvtable[i][1] .. "}"
    .. "\\phantomsection\\addcontentsline{toc}{subsubsection}{Synthèse}"
    .. "{\\LARGE\\faUser~| " .. csvtable[i][1]
    .. "\\hfill  ".. TitreFichePositionnement .." \\hfill "..Niveau.."}"  
    .. "\\\\"
    .. "\\vspace{0.5 em}\\\\"


    FicheRemediation = "\\phantomsection\\addcontentsline{toc}{subsubsection}{Remédiation}"
    .. "{\\LARGE\\faUser~| " .. csvtable[i][1]
    .. "\\hfill  ".. TitreFicheRemediation .."  \\hfill "..Niveau.."}"
    .. "\\\\"
    .. "\\vspace{0.5 em}\\\\"

    -- Notation
    FicheEleveTex = FicheEleveTex
    .. "{\\Large Notation"
    local Note = TabNotesExercices[i-RefPtMax+2][#TabNotesExercices[1]]
    local PtsMax= TabNotesExercices[2][#TabNotesExercices[1]]
    if PtsMax%1==0 then PtsMax=math.floor(TabNotesExercices[2][#TabNotesExercices[1]]) end
    if  tonumber(Note) ~= nil then
                  if NoteEnPourcentage == true then
                    NoteAffiche=math.floor(TabNotesExercices[i-RefPtMax+2][#TabNotesExercices[1]]/TabNotesExercices[2][#TabNotesExercices[1]]*100+0.5)
                else
                  NoteAffiche=TabNotesExercices[i-RefPtMax+2][#TabNotesExercices[1]]
                end
    else
      NoteAffiche=LettreAbs
    end
        
    -- NoteAffiche=TabNotesExercices[i-RefPtMax+2][#TabNotesExercices[1]]
    if tonumber(NoteAffiche)~=nil then 
              if NoteEnPourcentage==true then
      FicheEleveTex = FicheEleveTex .. "\\hrulefill ~ \\circled{\\np{".. NoteAffiche .."}\\%}}" 
                else FicheEleveTex = FicheEleveTex .. "\\hrulefill ~ \\circled{\\np{".. NoteAffiche .."}/\\np{".. PtsMax.."}}}" end
    else
      FicheEleveTex = FicheEleveTex .. "\\hrulefill ~ \\circled{".. NoteAffiche .."}}" 
    end

    FicheEleveTex = FicheEleveTex.. "\\begin{tasks}[label=$\\empty$,item-indent=15pt](4)"
    for ColTabNotesExercices=2,#TabNotesExercices[1]-1 do
        local Note = TabNotesExercices[i-RefPtMax+2][ColTabNotesExercices]
        if tonumber(Note)~=nil then
            if Note % 1 == 0 then
              Note = tostring(Note):gsub("%.0", "")
            end
            Note = "\\np{"..Note.."}"
            PtsMax = TabNotesExercices[2][ColTabNotesExercices]
            if PtsMax % 1 == 0 then
              PtsMax = tostring(PtsMax):gsub("%.0", "")
            end
            PtsMax = "\\np{"..PtsMax.."}"
            FicheEleveTex = FicheEleveTex .. "\\task Exercice" .. TabNotesExercices[1][ColTabNotesExercices] .. ": " 
            .. Note
            .. "/"
            .. PtsMax
        end

    end

    FicheEleveTex = FicheEleveTex .. "\\end{tasks}" 

    -- Compétences
    FicheEleveTex = FicheEleveTex .. "{\\Large Compétences}\\\\"
    for j=2,#TabComp[1] do
      NomCompetence = TabComp[1][j]
      NivComp = TabComp[i-RefPtMax+1][j]

      -- if type(NivComp)~="number" then NivComp=LettreAbs end

      FicheEleveTex = FicheEleveTex 
 
      .. "{\\large " .. NomCompetence .. "} \\hrulefill ~{\\Large \\circled{" .. NivComp .. " }}\\\\" 
      .. "\\vspace{-2em}"
      .. "\\begin{tasks}[label=$\\empty$,item-indent=15pt](2)"
        for LigneComp=4,RefPtMax-1 do
          for ColComp=2,#csvtable[1] do 
            if NomCompetence==csvtable[LigneComp][1] then

    if  tonumber(csvtable[LigneComp][ColComp])~=nil then
                  if math.floor(csvtable[LigneComp][ColComp])==1 then
                    if csvtable[2][ColComp] ~= '' then
                      FicheEleveTex = FicheEleveTex .. "\\task Ex" .. csvtable[1][ColComp] .. "." .. csvtable[2][ColComp] .. "\\quad "
                    else
                        FicheEleveTex = FicheEleveTex .. "\\task Ex" .. csvtable[1][ColComp] .. "\\quad "
                    end

                    FicheEleveTex = FicheEleveTex ..  csvtable[3][ColComp] .. "\\dotfill "
                    Note = csvtable[i][ColComp]
                    if tonumber(Note)~=nil then
                      Note = "\\np{"..Note.."}"
                    else
                      Note = LettreAbs
                    end
                    PtsMax = csvtable[RefPtMax][ColComp]
                    PtsMax = "\\np{"..PtsMax.."}"
                    FicheEleveTex = FicheEleveTex .. Note .. "/" .. PtsMax
                  end
      end


            end
          end
        
      end
      
      FicheEleveTex = FicheEleveTex 
      .. "\\end{tasks}"

 
  if tonumber(NivComp)~=nil  then
    if NivComp==1 or NivComp==2 then
      FicheRemediation = FicheRemediation .. "\\subfile{./Remediation/" .. NomCompetence .. "-niv1.tex}\\par"
    else
      FicheRemediation = FicheRemediation .. "\\subfile{./Remediation/" .. NomCompetence .. "-niv2.tex}\\par"
    end
  else
  FicheRemediation = FicheRemediation .. "\\subfile{./Remediation/" .. NomCompetence .. "-niv1.tex}\\par"
  end

  end

    FicheRemediation=FicheRemediation.. "\\newpage"
    FicheEleveTex = FicheEleveTex .. "\\newpage"
    
    if afficheFichePosition then
      tex.sprint(FicheEleveTex)
    end
    tex.sprint()
    if afficheFicheRemediation then
      tex.sprint(FicheRemediation)
    end
  end
end


  function ExportPronote(TabComp,TabNotesExercices)
  local Export = {}
  for i=2,#TabNotesExercices do
        -- Créer une nouvelle ligne pour Export
        local newRow = {}

        -- Ajouter la première et la dernière colonne de TabNotesExercices
        -- if tonumber(TabNotesExercices[i][#TabNotesExercices[i]])~=nil then

        -- if NoteEnPourcentage==true then
        -- local NoteAffiche = math.floor(TabNotesExercices[i][#TabNotesExercices[i]]/TabNotesExercices[2][#TabNotesExercices[2]]*100+0.5)
      -- else NoteAffiche = TabNotesExercices[i][#TabNotesExercices[i]] end
      local NotePronote = TabNotesExercices[i][#TabNotesExercices[i]]
      -- else
      -- local NoteAffiche = LettreAbs
      -- end
        table.insert(newRow, TabNotesExercices[i][1]) -- Première colonne
        table.insert(newRow,NotePronote) -- Dernière colonne

        -- Ajouter les colonnes de TabComp (de la 2ème à l'avant-dernière)
        for j = 2, #TabComp[1] - 1 do
            table.insert(newRow, TabComp[i-1][j])
        end

        -- Ajouter la nouvelle ligne au tableau Export
        table.insert(Export, newRow)
      end

      local cheminFichier = "ExportPronote.csv"

        -- Ouvrir le fichier en mode écriture
        local fichier = io.open(cheminFichier, "w")
        for i, ligne in ipairs(Export) do
            local valeurs = {}

            for j, valeur in ipairs(ligne) do
                if type(valeur) == "number" then
                    -- Remplacer le point par une virgule pour les nombres
                    valeur = tostring(valeur):gsub("%.", ",")
                end
                table.insert(valeurs, valeur)
            end
            -- Concaténer les valeurs avec des point-virgules
            local ligneCSV = table.concat(valeurs, ";")
            -- Écrire la ligne dans le fichier
            fichier:write(ligneCSV .. "\n")
        end
        -- Fermer le fichier
        fichier:close()
        return Export
  end
------------------------------------------------------------------------------------------------
  ---- Corps du programme
  csvtable = csvtotable(NomFichierCSV)
  -- AfficheTab(csvtable)
  TabComp=ExtractComp(csvtable)
  --  AfficheTab(TabComp)
  TabNotesExercices=ExtractNotesExercices(csvtable)
-- AfficheTab(TabNotesExercices)

ExportPronote=ExportPronote(TabComp,TabNotesExercices)
--  AfficheTab(ExportPronote)
 CreateFichesSynthese(csvtable,TabComp,TabNotesExercices)


\end{luacode*}


%Le texte du document




\end{document}