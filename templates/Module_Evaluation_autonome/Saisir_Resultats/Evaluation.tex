\documentclass{article}

\usepackage{geometry}
\usepackage{luacode}
\usepackage[np]{numprint}
\usepackage{fontawesome5}
\usepackage{tikz}
\usepackage{amsmath}
\usepackage{amsfonts}
\usepackage{mathrsfs}	
\usepackage{tasks}
\usepackage{bookmark}
\usepackage{subfiles}

\geometry{
    paperwidth=21cm
    ,paperheight=29.7cm
    ,top=1cm
    ,bottom=1cm
    ,right=1cm
    ,left=1cm
    ,includehead
    ,includefoot
    ,marginparsep=0cm
    ,marginparwidth=0cm
    ,headheight=0cm
    ,headsep=0cm
    ,footskip=0cm
}

\AtBeginDocument{\npproductsign{\times}\npthousandsep{\,}\npthousandthpartsep{\,}}

\pagestyle{empty}

% \renewenvironment{correction}[1][Correction]{\textbf{#1} \\}{}

\newcommand*\circled[1]{\tikz[baseline=(char.base)]{
            \node[shape=circle,draw,inner sep=2pt] (char) {#1};}}

\setlength\parindent{0pt}

\begin{document}


%%%%%%%  Contenu du document %%%%%%%%

    
    \begin{luacode*}
    nomDS = "\\LARGE{DS}"
    communiquer = ""
    calculer = ""
    representer = ""
    raisonner = ""
    chercher = ""
    modeliser = ""
    compteurcompetence = 0
    
    function csvtotable(filename)
        local fichier = io.open(filename, "r")
        local csvtable = {}
        for ligne in fichier:lines() do
            local lignecsvtable = {}
            for value in ligne:gmatch("[^;]+") do
                value = value:gsub(",", ".")  -- Remplacer les virgules par des points
                table.insert(lignecsvtable, value) 
            end
            table.insert(csvtable, lignecsvtable) 
        end
        fichier:close()
        return csvtable
    end
    
    
    
    function extraction_globale(csvtable,number)
       local maxlignes=#csvtable
       local longueur = csvtable[1]
       local maxcolonnes = #longueur
       for j=2,maxcolonnes do
          tex.sprint(csvtable[1][j])
       end
    end
    
    --1:Exercice
    --2:Question
    --3:Description
    --4:Calculer
    --5:Chercher
    --6:Communiquer
    --7:Modéliser
    --8:Raisonner
    --9:Représenter
    
    --10:Points max
    --10+:Points élève
    
    function extraction_competence(csvtable,C,E) --C:numéro de compétence ; E:numéro de l'élève à extraire
        local maxlignes=#csvtable
        local longueur = csvtable[1]
        local maxcolonnes = #longueur
    
        local points = 0
        local pointsmax=0
        local competence=nil
        local ratio=nil
        local question=nil
        local liste_score={}
    
    
        --Attribution globale de la compétence
        for j=2,maxcolonnes do
            if csvtable[C][j] == "1" then --On vérifie si la compétence est à 1 colonne par colonne
                if csvtable[2][j]~="Bonus" then
                    pointsmax = pointsmax + csvtable[10][j]
                end
                if csvtable[E][j] ~= "a" then
                    points = points + csvtable[E][j]
    
                    --Récupération du score par exercice et par question
                    local pts = csvtable[E][j]
                    local ptsmax = csvtable[10][j]
    
                    if pts%1 == 0 then
                        pts = math.floor(pts)
                    end
                    if ptsmax%1 == 0 then
                        ptsmax = math.floor(ptsmax)
                    end            
    
                    -- construction des item competences
                    question = "$\\underset{\\text{".. csvtable[3][j] .."}}{\\text{Ex " .. csvtable[1][j] .."-" .. csvtable[2][j] .. "}: \\mathbf{\\np{" .. pts .. "}/\\np{" .. ptsmax .. "}}}$"
    
                    table.insert(liste_score,question)
                end
            end
        end
    
    
        --Si la compétence est sollicitée, attribution d'un niveau
        if pointsmax ~= 0 then
            if csvtable[E][2] ~= "a" then
                ratio = points/pointsmax
                if ratio >= 0.8 then
                    competence = 4
                elseif ratio >= 0.6 then
                    competence = 3
                elseif ratio >= 0.3 then
                    competence = 2
                else
                    competence = 1
                end
            else
                competence = "a"
            end
    
    
        end
        --La fonction retourne le score global de la compétence, et la liste des scores à chacune des questions
        return competence,liste_score
    end
    
    function extraction_notes(csvtable,E)
        local maxlignes=#csvtable
        local longueur = csvtable[1]
        local maxcolonnes = #longueur
    
        local points = 0
        local pointsmax=0
        local total = 0
        local totalmax = 0
    
        local exercice = 1
        local question = nil
        local liste_score={}
        local pourcentage=0
    
        if csvtable[E][2] ~= "a" then
            for j=2,maxcolonnes do
            
                if csvtable[1][j] ~= tostring(exercice) then --On vérifie si on est toujours dans le bon exercice
    
                    if points%1 == 0 then
                    points = math.floor(points)
                    end
    
                    if pointsmax%1 == 0 then
                        pointsmax=math.floor(pointsmax)
                    end
    
                    question = "Exercice " .. exercice .. " : $\\mathbf{\\np{" .. points .. "}/\\np{" .. pointsmax .. "}}$"
                    table.insert(liste_score,question)
                    exercice = exercice + 1
                    points = 0
                    pointsmax = 0
                end
    
                points = points + csvtable[E][j]
                if csvtable[2][j]~="Bonus" then
                    pointsmax = pointsmax + csvtable[10][j]
                end
    
                total = total + csvtable[E][j]
                if csvtable[2][j]~="Bonus" then
                    totalmax = totalmax + csvtable[10][j]
                end
            end
    
            --Insertion du dernier exercice
            if points%1 == 0 then
                points = math.floor(points)
            end
    
            if pointsmax%1 == 0 then
                pointsmax=math.floor(pointsmax)
            end
    
            question = "Exercice " .. exercice .. " : $\\mathbf{\\np{" .. points .. "}/\\np{" .. pointsmax .. "}}$"
            table.insert(liste_score,question)
    
            pourcentage = math.floor(total/totalmax*100)
        else
            pourcentage = "a"
        end
    
        return pourcentage,liste_score
    end
    
    function affiche_eleve(csvtable,E)
       
        local ligne_nom = ""
        local ligne_exercice = ""
        local ligne_competence = ""
        local recap_eleve = {}
        local recap_classe = {}
        local entete_classe = {"ÉLÈVE","POURCENTAGE"}
    
        --Création de la première ligne : NOM Prénom et pourcentage
        pourcentage,liste_score = extraction_notes(csvtable,E)
        if pourcentage ~= "a" then
            ligne_nom = "\\phantomsection\\addcontentsline{toc}{subsection}{" .. csvtable[E][1] .. "}"
            ligne_nom = ligne_nom .. "\\Large\\faUserGraduate\\, " .. csvtable[E][1] .. "\\hfill " .. nomDS .. " \\hfill \\circled{" .. pourcentage .. "\\%}"
        end
        recap_eleve = {csvtable[E][1],pourcentage}
        -- recap_eleve = recap_eleve,pourcentage
    
        --Création de la deuxième ligne avec le récapitulatif par exercice
        if pourcentage ~= "a" then
            for i=1,#liste_score do
            ligne_exercice = ligne_exercice .. "\\task \\large " .. liste_score[i]
            end
    
            ligne_exercice = "\\normalsize\\begin{tasks}[label=$\\empty$,item-indent=15pt](4) " .. ligne_exercice .. "\\end{tasks}"
        end
    
        --Création des lignes pour chaque compétence
        function ligne_competence(csvtable,C,E)
            local competence = nil
            local liste_score = {}
            local ligne_competence = ""
            local recap_eleve = {}
            local entete_classe = {}
            competence,liste_score = extraction_competence(csvtable,C,E)
            if competence then
                entete_classe = csvtable[C][1]
                recap_eleve = competence
                if competence ~= "a" then
                    ligne_competence = "\\begin{minipage}{0.25\\linewidth}\\bfseries\\Large " .. csvtable[C][1] .. "\\end{minipage}\\hfill\\begin{minipage}{0.05\\linewidth}\\circled{" .. competence .. "}\\end{minipage}\\hfill\\begin{minipage}{0.7\\linewidth}\\begin{tasks}[label=$\\empty$,item-indent=15pt](4) "
    
                        if csvtable[C][1]=="Communiquer" then
                            if competence < 3 then
                                communiquer = "\\subfile{../Exercices_Remediation/communiquer_1.tex}"
                            else
                                communiquer = "\\subfile{../Exercices_Remediation/communiquer_2.tex}"
                            end
    
                        elseif csvtable[C][1]=="Calculer" then
                            if competence < 3 then
                                calculer = "\\subfile{../Exercices_Remediation/calculer_1.tex}"
                            else
                                calculer = "\\subfile{../Exercices_Remediation/calculer_2.tex}"
                            end
    
                        elseif csvtable[C][1]=="Représenter" then
                            if competence < 3 then
                                representer = "\\subfile{../Exercices_Remediation/representer_1.tex}"
                            else
                                representer = "\\subfile{../Exercices_Remediation/representer_2.tex}"
                            end
                       
                        elseif csvtable[C][1]=="Raisonner" then
                            if competence < 3 then
                                raisonner = "\\subfile{../Exercices_Remediation/raisonner_1.tex}"
                            else
                                raisonner = "\\subfile{../Exercices_Remediation/raisonner_2.tex}"
                            end
    
                        elseif csvtable[C][1]=="Chercher" then
                            if competence < 3 then
                                chercher = "\\subfile{../Exercices_Remediation/chercher_1.tex}"
                            else
                                chercher = "\\subfile{../Exercices_Remediation/chercher_2.tex}"
                            end
    
                        elseif csvtable[C][1]=="Modéliser" then
                            if competence < 3 then
                                modeliser = "\\subfile{../Exercices_Remediation/modeliser_1.tex}"
                            else
                                modeliser = "\\subfile{../Exercices_Remediation/modeliser_2.tex}"
                            end
                        end
                    
                        for i=1,#liste_score do
                        ligne_competence = ligne_competence .. "\\task " .. liste_score[i]
                        end
    
                    ligne_competence = ligne_competence .. "\\end{tasks}\\end{minipage}\\par\\par"
                end
                return ligne_competence,recap_eleve,entete_classe
            end
            
        end
            if pourcentage ~= "a" then
                tex.sprint("\\begin{minipage}{\\linewidth}")
                tex.sprint(ligne_nom)
                tex.sprint("\\vspace{0.2cm}\\par\\hrule\\par")
                tex.sprint("\\large\\fbox{Détail par exercice}\\par\\vspace{0.2cm}")
                tex.sprint(ligne_exercice)
                tex.sprint("\\vspace{0.2cm}\\par\\hrule\\par")
                tex.sprint("\\large\\fbox{Détail par compétence}\\normalsize\\par")
                tex.sprint("\\vspace{0.2cm}")
            end
            for i=4,9 do
            local ligne
            local eleve
            local entete
            ligne,eleve,entete = ligne_competence(csvtable,i,E,recap_eleve)
            table.insert(recap_eleve,eleve)
            table.insert(entete_classe,entete)
                if ligne then
                    tex.sprint(ligne)
                    tex.sprint("\\vspace{0.4cm}")
                end
            end
                if pourcentage ~= "a" then
            tex.sprint("\\vspace{0.2cm}\\hrule")
            tex.sprint{"\\end{minipage}\\par"}
            tex.sprint{"\\vspace{0.5em}"}
            end
            -- tex.sprint("\\par\\vspace{0.5em}\\documentclass[../Evaluation.tex]{subfiles}

\begin{document}


\end{document} \\par")     
            tex.sprint(communiquer .. "\\par")    
            tex.sprint(calculer .. "\\par")   
            tex.sprint(representer .. "\\par")   
            tex.sprint(raisonner .. "\\par")   
            tex.sprint(chercher .. "\\par")   
            tex.sprint(modeliser .. "\\par")   
        return recap_eleve,entete_classe
    end
    
    function genere(csvtable)
        local recap_eleve = {}
        local recap_classe = {}
        local entete_classe = {}
    
        for i =11,#csvtable do
           
            recap_eleve,entete_classe = affiche_eleve(csvtable,i)
    
            tex.sprint("\\newpage")
            
            if i == 11 then
                table.insert(recap_classe,entete_classe)
            end
            table.insert(recap_classe,recap_eleve)
        end
        return recap_classe
    end
    
    filename = "Export_pronote.csv"
    csvtable = csvtotable("Evaluation.csv")
    recap_classe = genere(csvtable)
    
        local file = assert(io.open(filename, "w"))
    
        function escape_csv_field(field)
            if string.find(field, '[,"]') then
                field = '"' .. string.gsub(field, '"', '""') .. '"'
            end
            return field
        end
    
        for _, row in ipairs(recap_classe) do
            local csv_row = {}
            for _, field in ipairs(row) do
                table.insert(csv_row, escape_csv_field(tostring(field)))
            end
            file:write(table.concat(csv_row, ",") .. "\n")
        end
    
        file:close()
    
    
    
    
    \end{luacode*}
     

\end{document}