\documentclass[../Chapitre2.tex]{subfiles}

\begin{document}


\definirchapitre{2}{Nombres relatifs et repérage}
\definirniveau{5}

\begin{luacode}
  mesParametres('Cours')
\end{luacode}
\parametrage


\partie{Premières définitions}

\souspartie{Nombres positifs et négatifs}

%%%%%%%%%%%%%%%%%%%%
%%%% DÉFINITION %%%%
%%%%%%%%%%%%%%%%%%%%

\begin{definition}[Définitions]

\begin{itemize}
  \item Un \important{nombre positif} est un nombre supérieur ou égal à zéro.
  \item Un \important{nombre négatif} est un nombre inférieur ou égal à zéro.
  \item Un \important{nombre relatif} est un nombre positif ou négatif.
\end{itemize}

\end{definition}

%%%%%%%%%%%%%%%%%
%%%% EXEMPLE %%%%
%%%%%%%%%%%%%%%%%

\begin{exemple}
    \begin{itemize}
      \item $5$ ; $+18$ ; $0$ sont des nombres positifs.
      \item $-17$ ; $-489$ ; $0$ sont des nombres négatifs.
    \end{itemize}
\end{exemple}

%%%%%%%%%%%%%%%%%%
%%%% REMARQUE %%%%
%%%%%%%%%%%%%%%%%%

\begin{remarque}[Remarques]
  \begin{itemize}
    \item $0$ est à la fois positif et négatif.
    \item Le signe $+$ n'est pas toujours écrit pour les nombres postifs : $+12$ peut s'écrire simplement $12$.
  \end{itemize}
\end{remarque}

\souspartie{Opposé d'un nombre}

%%%%%%%%%%%%%%%%%%%%
%%%% DÉFINITION %%%%
%%%%%%%%%%%%%%%%%%%%

\begin{definition}
  Deux nombres sont dits \important{opposés} lorsqu'ils ont la même valeur mais des signes différents.
\end{definition}

%%%%%%%%%%%%%%%%%
%%%% EXEMPLE %%%%
%%%%%%%%%%%%%%%%%

\begin{exemple}
    \begin{itemize}
      \item $+5$ et $-5$ sont opposés.
      \item L'opposé de $-8$ est $+8$.
    \end{itemize}
\end{exemple}

\partie{Repérage}

\souspartie{Sur une droite graduée}

%%%%%%%%%%%%%%%%%
%%%% MÉTHODE %%%%
%%%%%%%%%%%%%%%%%

\begin{remarque}
    Un nombre relatif peut se représenter sur une droite graduée.
\end{remarque}

%%%%%%%%%%%%%%%%%
%%%% EXEMPLE %%%%
%%%%%%%%%%%%%%%%%

\begin{exemple}
  \Reperage[AffichageNom,AffichageGrad,ValeurMin=-5,ValeurMax=5]{-5/,2/B,-3/A,5/}

  \bigskip 

  On note les abscisses des points $A(-3)$ et $B(2)$.
\end{exemple}

\sautfiche



\partie{Dans un repère}

%%%%%%%%%%%%%%%%%%%%
%%%% DÉFINITION %%%%
%%%%%%%%%%%%%%%%%%%%

\begin{definition}
  Un repère permet d'identifier chaque point du plan par deux coordonnées. 
  
  On note $A(x;y)$ un point du plan, où :

  \begin{itemize}
    \item $x$ représente son \important{abscisse}, obtenue horizontalement ($\rightarrow$)
    \item $y$ représente son \important{ordonnée}, obtenue verticalement ($\uparrow$)
  \end{itemize}
\end{definition}

%%%%%%%%%%%%%%%%%
%%%% EXEMPLE %%%%
%%%%%%%%%%%%%%%%%

\begin{exemple}
\adjustbox{valign=t}{\begin{minipage}{0.48\linewidth}%
    \Reperage[Plan,AffichageNom,LectureCoord,AffichageGrad]{1/2/A,-2/1/B,2/-3/C,-4/-1/D}
\end{minipage}}\hfill%
\adjustbox{valign=t}{\begin{minipage}{0.48\linewidth}%
    On peut lire les coordonnées suivantes :
  
    \begin{itemize}
      \item $A(1\,;\,2)$
      \item $B(-2\,;\,1)$
      \item $C(2\,;\,-3)$
      \item $D(-4\,;\,-1)$
    \end{itemize}

\end{minipage}}%
\end{exemple}

\end{document}