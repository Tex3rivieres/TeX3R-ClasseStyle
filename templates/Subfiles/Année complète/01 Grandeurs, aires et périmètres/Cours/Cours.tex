\documentclass{classe-tex3R}
\usepackage{style-tex3R}

\begin{document}

\definirchapitre{1}{Grandeurs, aires et périmètre}
\definirniveau{5}

\begin{luacode}
  mesParametres('Cours')
  
\end{luacode}
\parametrage

\subsection{Grandeurs et unités}

%%%%%%%%%%%%%%%%%%%%
%%%% DÉFINITION %%%%
%%%%%%%%%%%%%%%%%%%%

\begin{definition}
  \begin{itemize}
    \item   Une \important{grandeur} est quelque chose que l'on peut mesurer.
    \item Une \important{unité} permet de quantifier une grandeur.
  \end{itemize}
\end{definition}

%%%%%%%%%%%%%%%%%
%%%% EXEMPLE %%%%
%%%%%%%%%%%%%%%%%

\begin{exemple}
    \begin{itemize}
      \item Il fait chaud. La grandeur est la température, elle se mesure en degré Celsius ($^\circ$C).
      \item Je suis plus petit que mon ami. La grandeur est la taille, elle se mesure en mètres (m).
      \item Je m'ennuie, ce spectacle est trop long. La grandeur est le temps, il se mesure en heures (h).
    \end{itemize}
\end{exemple}

\sautdiapo



\subsection{Périmètre d'une figure}

%%%%%%%%%%%%%%%%%%%%
%%%% DÉFINITION %%%%
%%%%%%%%%%%%%%%%%%%%

\begin{definition}
  Le \important{périmètre} d'une figure est la longueur de son contour. Il se mesure avec les multiples et sous-multiples du mètre (m, cm, km par exemple).
\end{definition}

\sautdiapo

%%%%%%%%%%%%%%%%%%%
%%%% PROPRIÉTÉ %%%%
%%%%%%%%%%%%%%%%%%%

\begin{propriete}[s (constatées)]
  Les formules de périmètre des figures usuelles sont :

\adjustbox{valign=t}{\begin{minipage}{0.32\linewidth}%
      \Formule[Perimetre,Surface=polygone,Largeur=6cm,Ancre={(3.5,-2)}]
\end{minipage}}\hfill%
\adjustbox{valign=t}{\begin{minipage}{0.32\linewidth}%
      \Formule[Perimetre,Surface=cercle,Largeur=6cm,Ancre={(3,-2)}]
\end{minipage}}\hfill%
\adjustbox{valign=t}{\begin{minipage}{0.32\linewidth}%
      % \Formule[Perimetre,Surface=polygone,Largeur=4cm,Ancre={(3,-8)}]
\end{minipage}}%

\vspace{4cm}



\end{propriete}

%%%%%%%%%%%%%%%%%
%%%% EXEMPLE %%%%
%%%%%%%%%%%%%%%%%

\begin{exemple}
\vspace{-3cm}

\psset{xunit=1.5cm,yunit=1cm,algebraic=true,dimen=middle,dotstyle=+,dotsize=10pt 0,linewidth=2pt,arrowsize=3pt 2,arrowinset=0.25}
\begin{pspicture*}(-0.2786494928188068,-5.249755414029417)(13.373264352162817,7.269600555920095)
\psline[linewidth=2pt](1,1)(2,3)
\psline[linewidth=2pt](2,3)(4,2)
\psline[linewidth=2pt](4,2)(5,1)
\psline[linewidth=2pt](5,1)(4,0)
\psline[linewidth=2pt](4,0)(3,1)
\psline[linewidth=2pt](3,1)(1,1)
\pscircle[linewidth=2pt](9.73349604616283,1.4732417016402937){2.2978017225859303}
\psline[linewidth=2pt](9.73349604616283,1.4732417016402937)(8.592590078685769,2.9898118291402893)
% \begin{scriptsize}
\psdots[dotsize=0pt 0,dotstyle=x](1,1)
\rput[bl](0.7314696930206407,0.5354726503237804){$A$}
\psdots[dotsize=0pt 0,dotstyle=x](2,3)
\rput[bl](1.8640275680527483,3.152599631816894){$B$}
\rput[bl](0.8222615798532086,2.1730901182756117){$\np[cm]{2.2}$}
\psdots[dotsize=0pt 0,dotstyle=x](4,2)
\rput[bl](4.022009465073386,2.111870773679281){$C$}
\rput[bl](2.996585443084856,2.678149711195335){$\np[cm]{2.2}$}
\psdots[dotsize=0pt 0,dotstyle=x](5,1)
\rput[bl](5.215786684701824,0.7344355202618534){$D$}
\rput[bl](4.542373894142192,1.5762015084613927){$\np[cm]{1.4}$}
\psdots[dotsize=0pt 0,dotstyle=x](4,0)
\rput[bl](3.9454852843279733,-0.4134271909193366){$E$}
\rput[bl](4.5882884025894395,0.24468076349121232){$\np[cm]{1.4}$}
\psdots[dotsize=0pt 0,dotstyle=x](3,1)
\rput[bl](2.9812806069357736,1.1476660962870817){$F$}
\rput[bl](3.0863401998554973,-0.0426065033673581){$\np[cm]{1.4}$}
\rput[bl](1.8793324042018307,0.6273016672182756){$\np[cm]{2}$}
\psdots[dotstyle=+](9.73349604616283,1.4732417016402937)
\rput[bl](9.791932693277504,1.6221160169086404){$G$}
\psdots[dotstyle=+](8.592590078685769,2.9898118291402893)
\rput[bl](8.359374818245395,3.137294795667811){$H$}
\rput[bl](8.359374818245395,-0.537294795667811){$\mathcal{C}$}
\rput[bl](9.34809244495411,2.2649191351701066){$\np[cm]{1.9}$}
% \end{scriptsize}
\end{pspicture*}

\vspace{-4em}
  
  \adjustbox{valign=t}{\begin{minipage}{0.56\linewidth}%
  
  \begin{align*}
    \mathcal{P}_{ABCDEF} & = AB + BC + CD + DE + EF + FA\\
    & = \np[cm]{2.2} + \np[cm]{2.2} + \np[cm]{1.4} + \np[cm]{1.4} + \np[cm]{1.4} + \np[cm]{2}\\
    &= \np[cm]{10.6}\\
  \end{align*}
  
  Le périmètre de $ABCDEF$ est de $\np[cm]{10.6}$.
  \end{minipage}}\hfill%
  \adjustbox{valign=t}{\begin{minipage}{0.4\linewidth}%
    
    \begin{align*}
      \mathcal{P}_\text{Cercle}& = \mathrm{diamètre} \times\pi\\
      & = \np[cm]{1.9}\times 2 \times \pi\\
      &= 3,8\pi\,\text{cm}\\
      &\approx \np[cm]{11.94}\\
    \end{align*}
    
    Le périmètre de $\mathcal{C}$ est de $\np{3,8}\pi\,\text{cm}$ soit environ $\np[cm]{11.94}$.
  \end{minipage}}%
\end{exemple}



%%%%%%%%%%%%%%%%%
%%%% MÉTHODE %%%%
%%%%%%%%%%%%%%%%%

\begin{methode}
  Pour convertir une longueur dans une autre unité, on peut utiliser le tableau suivant :

\sautdeligne

  \Tableau[Metre,NbLignes=1,Fleches]{}

\sautdeligne

\end{methode}

%%%%%%%%%%%%%%%%%
%%%% EXEMPLE %%%%
%%%%%%%%%%%%%%%%%

\sautdiapo

\begin{exemple}
Pour placer $\np[dm]{412.5}$ dans le tableau, on repère son chiffre des unités (2) et on le place dans la colonne de son unité (dm) puis on place les autres chiffres, un par colonne.

  \Tableau[Metre,NbLignes=1]{004125/1}

  On peut lire ici que $\np[dm]{412.5} =$ $ \np[cm]{4125} = \np[dam]{4.125} = \np[km]{0.04125}$

\end{exemple}

\sautfiche


\subsection{Aire d'une figure}

%%%%%%%%%%%%%%%%%%%%
%%%% DÉFINITION %%%%
%%%%%%%%%%%%%%%%%%%%

\begin{definition}
  L'\important{aire} d'une figure est la mesure de sa surface intérieure. Elle se mesure avec les multiples et sous-multiples du mètre carré ($\text{m}^2$, $\text{cm}^2$, $\text{km}^2$ par exemple).
\end{definition}


%%%%%%%%%%%%%%%%%%%
%%%% PROPRIÉTÉ %%%%
%%%%%%%%%%%%%%%%%%%

\begin{propriete}[s (constatées)]
  Les formules d'aire des figures usuelles sont :

\adjustbox{valign=t}{\begin{minipage}{0.32\linewidth}%
      \Formule[Aire,Surface=carre,Largeur=5cm,Ancre={(3,-2)}]
\end{minipage}}\hfill%
\adjustbox{valign=t}{\begin{minipage}{0.32\linewidth}%
      \Formule[Aire,Surface=rectangle,Largeur=5cm,Ancre={(3,-2)}]
\end{minipage}}\hfill%
\adjustbox{valign=t}{\begin{minipage}{0.32\linewidth}%
      \Formule[Aire,Surface=disque,Largeur=5cm,Ancre={(3,-2)}]
\end{minipage}}%

\adjustbox{valign=t}{\begin{minipage}{0.32\linewidth}%
      \Formule[Aire,Surface=parallelogramme,Largeur=6cm,Ancre={(3.5,-6)}]
\end{minipage}}\hfill%
\adjustbox{valign=t}{\begin{minipage}{0.32\linewidth}%
      \Formule[Aire,Surface=triangle,Largeur=7cm,Ancre={(3,-6)}]
\end{minipage}}\hfill%
\adjustbox{valign=t}{\begin{minipage}{0.32\linewidth}%
      % \Formule[Perimetre,Surface=polygone,Largeur=4cm,Ancre={(3,-8)}]
\end{minipage}}%

\vspace{8cm}
\end{propriete}

%%%%%%%%%%%%%%%%%
%%%% MÉTHODE %%%%
%%%%%%%%%%%%%%%%%

\begin{methode}
  Pour convertir une aire dans une autre unité, on peut utiliser le tableau suivant :

\sautdeligne


    \Tableau[Carre,Colonnes,NbLignes=1,Fleches]{}


\sautdeligne

\end{methode}

\sautdiapo

%%%%%%%%%%%%%%%%%
%%%% EXEMPLE %%%%
%%%%%%%%%%%%%%%%%

\begin{exemple}
Pour placer $\np[dm^2]{4753.6}$ dans le tableau, on repère son chiffre des unités (3) et on le place dans la colonne la plus à droite de son unité ($\mathrm{dm}^2$) puis on place les autres chiffres, un par colonne.

\vspace{0.5em}

\Tableau[Carre,Colonnes,NbLignes=1]{0475360/6}

\vspace{1em}

    On peut lire ici que $\np[dm^2]{4753.6} =$ $ \np[dam^2]{0.47536} = \np[m^2]{47.536}$

\end{exemple}

\end{document}